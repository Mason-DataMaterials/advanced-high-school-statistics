\chapter{Probability}
\label{probability}

\index{probability|(}

Probability forms a foundation for statistics. You might already be familiar with many aspects of probability, however, formalization of the concepts is new for most. This chapter aims to introduce probability on familiar terms using processes most people have seen before.

\section{Defining probability}
\label{basicsOfProbability}

\begin{example}{A ``die'', the singular of dice, is a cube with six faces numbered \resp{1}, \resp{2}, \resp{3}, \resp{4}, \resp{5}, and \resp{6}. What is the chance of getting \resp{1} when rolling a die?}\label{probOf1}
If the die is fair, then the chance of a \resp{1} is as good as the chance of any other number. Since there are six outcomes, the chance must be 1-in-6 or, equivalently, $1/6$.
\end{example}

\begin{example}{What is the chance of getting a \resp{1} or \resp{2} in the next roll?}\label{probOf1Or2}
\resp{1} and \resp{2} constitute two of the six equally likely possible outcomes, so the chance of getting one of these two outcomes must be $2/6 = 1/3$.
\end{example}

\begin{example}{What is the chance of getting either \resp{1}, \resp{2}, \resp{3}, \resp{4}, \resp{5}, or \resp{6} on the next roll?}\label{probOf123456}
100\%. The outcome must be one of these numbers.
\end{example}

\begin{example}{What is the chance of not rolling a \resp{2}?}\label{probNot2}
Since the chance of rolling a \resp{2} is $1/6$ or $16.\bar{6}\%$, the chance of not rolling a \resp{2} must be $100\% - 16.\bar{6}\%=83.\bar{3}\%$ or $5/6$.

Alternatively, we could have noticed that not rolling a \resp{2} is the same as getting a \resp{1}, \resp{3}, \resp{4}, \resp{5}, or \resp{6}, which makes up five of the six equally likely outcomes and has probability $5/6$.
\end{example}

\begin{example}{Consider rolling two dice. If $1/6^{th}$ of the time the first die is a \resp{1} and $1/6^{th}$ of those times the second die is a \resp{1}, what is the chance of getting two \resp{1}s?}\label{probOf2Ones}
If $16.\bar{6}$\% of the time the first die is a \resp{1} and $1/6^{th}$ of \emph{those} times the second die is also a \resp{1}, then the chance that both dice are \resp{1} is $(1/6)\times (1/6)$ or $1/36$.
\end{example}

\subsection{Probability}

\index{random process|(}

We use probability to build tools to describe and understand apparent randomness. We often frame probability in terms of a \term{random process} giving rise to an \term{outcome}.
\begin{center}
\begin{tabular}{lll}
Roll a die &$\rightarrow$ & \resp{1}, \resp{2}, \resp{3}, \resp{4}, \resp{5}, or \resp{6} \\
Flip a coin &$\rightarrow$ & \resp{H} or \resp{T} \\
\end{tabular}
\end{center}
Rolling a die or flipping a coin is a seemingly random process and each gives rise to an outcome.

\begin{termBox}{\tBoxTitle{Probability}
The \term{probability} of an outcome is the proportion of times the outcome would occur if we observed the random process an infinite number of times.}
\end{termBox}

Probability is defined as a proportion, and it always takes values between 0~and~1 (inclusively). It may also be displayed as a percentage between 0\% and 100\%.

Probability can be illustrated by rolling a die many times. Consider the event ``roll a 1". The \term{relative frequency} of an event is the proportion of times the event occurs out of the number of trials. Let $\hat{p}_n$ be the proportion of outcomes that are \resp{1} after the first $n$ rolls. As the number of rolls increases, $\hat{p}_n$ (the relative frequency of rolls) will converge to the probability of rolling a \resp{1}, $p = 1/6$. Figure~\ref{dieProp} shows this convergence for 100,000 die rolls. The tendency of $\hat{p}_n$ to stabilize around $p$, that~is, the tendency of the relative frequency to stabilize around the true probability, is described by the \term{Law of Large Numbers}.

\begin{figure}[ht]
\centering
\includegraphics[width=0.85\textwidth]{ch_probability/figures/dieProp/dieProp}
\caption{The fraction of die rolls that are 1 at each stage in a simulation. The relative frequency tends to get closer to the probability $1/6 \approx 0.167$ as the number of rolls increases.}
\label{dieProp}
\end{figure}

\begin{termBox}{\tBoxTitle{Law of Large Numbers}
As more observations are collected, the observed proportion $\hat{p}_n$ of occurrences with a particular outcome after $n$ trials converges to the true probability $p$ of that outcome.}
\end{termBox}

Occasionally the proportion will veer off from the probability and appear to defy the Law of Large Numbers, as $\hat{p}_n$ does many times in Figure~\ref{dieProp}. However, these deviations become smaller as the number of rolls increases.

Above we write $p$ as the probability of rolling a \resp{1}. We can also write this probability as
\begin{eqnarray*}
P(\text{rolling a \resp{1}})
\end{eqnarray*}
\marginpar[\raggedright\vspace{-13mm}

$P(A)$\vspace{1mm}\\\footnotesize Probability of\\outcome $A$]{\raggedright\vspace{-13mm}

$P(A)$\vspace{1mm}\\\footnotesize Probability of\\outcome $A$}As we become more comfortable with this notation, we will abbreviate it further. For instance, if it is clear that the process is ``rolling a die'', we could abbreviate $P($rolling a \resp{1}$)$ as~$P($\resp{1}$)$.

\begin{exercise} \label{randomProcessExercise}
Random processes include rolling a die and flipping a coin. (a) Think of another random process. (b) Describe all the possible outcomes of that process. For instance, rolling a die is a random process with potential outcomes \resp{1}, \resp{2}, ...,~\resp{6}.~\footnote{Here are four examples. (i) Whether someone gets sick in the next month or not is an apparently random process with outcomes \resp{sick} and \resp{not}. (ii) We can \emph{generate} a random process by randomly picking a person and measuring that person's height. The outcome of this process will be a positive number. (iii) Whether the stock market goes up or down next week is a seemingly random process with possible outcomes \resp{up}, \resp{down}, and \resp{no\_\hspace{0.3mm}change}. Alternatively, we could have used the percent change in the stock market as a numerical outcome. (iv) Whether your roommate cleans her dishes tonight probably seems like a random process with possible outcomes \resp{cleans\_\hspace{0.3mm}dishes} and \resp{leaves\_\hspace{0.3mm}dishes}.}
\end{exercise}

What we think of as random processes are not necessarily random, but they may just be too difficult to understand exactly. The fourth example in the footnote solution to Guided Practice~\ref{randomProcessExercise} suggests a roommate's behavior is a random process. However, even if a roommate's behavior is not truly random, modeling her behavior as a random process can still be useful.

\begin{tipBox}{\tipBoxTitle{Modeling a process as random}
It can be helpful to model a process as random even if it is not truly random.}
\end{tipBox}

\index{random process|)}

\subsection{Disjoint or mutually exclusive outcomes}

\index{disjoint|(}
\index{mutually exclusive|(}

Two outcomes are called \term{disjoint} or \term{mutually exclusive} if they cannot both happen in the same trial. For instance, if we roll a die, the outcomes \resp{1} and \resp{2} are disjoint since they cannot both occur on a single roll. On the other hand, the outcomes \resp{1} and ``rolling an odd number'' are not disjoint since both occur if the outcome of the roll is a \resp{1}. The terms \emph{disjoint} and \emph{mutually exclusive} are equivalent and interchangeable.

Calculating the probability of disjoint outcomes is easy. When rolling a die, the outcomes \resp{1} and \resp{2} are disjoint, and we compute the probability that one of these outcomes will occur by adding their separate probabilities:
\begin{eqnarray*}
P(\text{\resp{1} or \resp{2}}) = P(\text{\resp{1}})+P(\text{\resp{2}}) = 1/6 + 1/6 = 1/3
\end{eqnarray*}
What about  the probability of rolling a \resp{1}, \resp{2}, \resp{3}, \resp{4}, \resp{5}, or \resp{6}? Here again, all of the outcomes are disjoint so we add the probabilities:
\begin{eqnarray*}
&&P(\text{\resp{1} or \resp{2} or \resp{3} or \resp{4} or \resp{5} or \resp{6}}) \\
	&&\quad= P(\text{\resp{1}})+P(\text{\resp{2}})+P(\text{\resp{3}})+P(\text{\resp{4}})+P(\text{\resp{5}})+P(\text{\resp{6}}) \\
	&&\quad= 1/6 + 1/6 + 1/6 + 1/6 + 1/6 + 1/6 = 1.
\end{eqnarray*}
The \term{Addition Rule} guarantees the accuracy of this approach when the outcomes are disjoint.

\begin{termBox}{\tBoxTitle{Addition Rule of disjoint outcomes} If $A_1$ and $A_2$ represent two disjoint outcomes, then the probability that one of them occurs is given by
\begin{eqnarray*}
P(A_1\text{ or } A_2) = P(A_1) + P(A_2)
\end{eqnarray*}
If there are many disjoint outcomes $A_1$, ..., $A_k$, then the probability that one of these outcomes will occur is
\begin{eqnarray}
P(A_1) + P(A_2) + \cdots + P(A_k)
\end{eqnarray}
}
\end{termBox}

\begin{exercise}
We are interested in the probability of rolling a \resp{1}, \resp{4}, or \resp{5}. (a) Explain why the outcomes \resp{1}, \resp{4}, and \resp{5} are disjoint. (b) Apply the Addition Rule for disjoint outcomes to determine $P($\resp{1} or \resp{4} or \resp{5}$)$.\footnote{(a) The random process is a die roll, and at most one of these outcomes can come up. This means they are disjoint outcomes. (b)~$P($\resp{1} or \resp{4} or \resp{5}$) = P($\resp{1}$)+P($\resp{4}$)+P($\resp{5}$) = \frac{1}{6} + \frac{1}{6} + \frac{1}{6} = \frac{3}{6} = \frac{1}{2}$}
\end{exercise}

\index{data!email|(}
\begin{exercise}
In the \data{email} data set in Chapter~\ref{summarizingData}, the \var{number} variable described whether no number (labeled \resp{none}), only one or more small numbers (\resp{small}), or whether at least one big number appeared in an email (\resp{big}). Of the 3,921 emails, 549 had no numbers, 2,827 had only one or more small numbers, and 545 had at least one big number. (a) Are the outcomes \resp{none}, \resp{small}, and \resp{big} disjoint? (b) Determine the proportion of emails with value \resp{small} and \resp{big} separately. (c) Use the Addition Rule for disjoint outcomes to compute the probability a randomly selected email from the data set has a number in it, small or big.\footnote{(a) Yes. Each email is categorized in only one level of \var{number}. (b) Small: $\frac{2827}{3921} = 0.721$. Big: $\frac{545}{3921} = 0.139$. (c) $P($\resp{small} or \resp{big}$) = P($\resp{small}$) + P($\resp{big}$) = 0.721 + 0.139 = 0.860$.}
\end{exercise}
\index{data!email|)}

\index{event|(}

Statisticians rarely work with individual outcomes and instead consider \indexthis{\emph{sets}}{sets} or \indexthis{\emph{collections}}{collections} of outcomes. Let $A$ represent the event where a die roll results in \resp{1} or \resp{2} and $B$~represent the event that the die roll is a \resp{4} or a \resp{6}. We write $A$ as the set of outcomes $\{$\resp{1},~\resp{2}$\}$ and $B=\{$\resp{4}, \resp{6}$\}$. These sets are commonly called \termsub{events}{event}. Because $A$ and $B$ have no elements in common, they are disjoint events. $A$ and $B$ are represented in Figure~\ref{disjointSets}.

\begin{figure}[hhh]
\centering
\includegraphics[height=0.7in]{ch_probability/figures/disjointSets/disjointSets}
\caption{Three events, $A$, $B$, and $D$, consist of outcomes from rolling a die. $A$ and $B$ are disjoint since they do not have any outcomes in common.}
\label{disjointSets}
\end{figure}

The Addition Rule applies to both disjoint outcomes and disjoint events. The probability that one of the disjoint events $A$ or $B$ occurs is the sum of the separate probabilities:
\begin{eqnarray*}
P(A\text{ or }B) = P(A) + P(B) = 1/3 + 1/3 = 2/3
\end{eqnarray*}

\begin{exercise}
(a) Verify the probability of event $A$, $P(A)$, is $1/3$ using the Addition Rule. (b) Do the same for event $B$.\footnote{(a) $P(A) = P($\resp{1} or \resp{2}$) = P($\resp{1}$) + P($\resp{2}$) = \frac{1}{6} + \frac{1}{6} = \frac{2}{6} = \frac{1}{3}$. (b) Similarly, $P(B) = 1/3$.}
\end{exercise}

\begin{exercise} \label{exerExaminingDisjointSetsABD}
(a) Using Figure~\ref{disjointSets} as a reference, what outcomes are represented by event $D$? (b) Are events $B$ and $D$ disjoint? (c) Are events $A$ and $D$ disjoint?\footnote{(a)~Outcomes \resp{2} and \resp{3}. (b)~Yes, events $B$ and $D$ are disjoint because they share no outcomes. (c)~The events $A$ and $D$ share an outcome in common, \resp{2}, and so are not disjoint.}
\end{exercise}

\begin{exercise}
In Guided Practice~\ref{exerExaminingDisjointSetsABD}, you confirmed $B$ and $D$ from Figure~\ref{disjointSets} are disjoint. Compute the probability that either event $B$ or event $D$ occurs.\footnote{Since $B$ and $D$ are disjoint events, use the Addition Rule: $P(B$ or $D) = P(B) + P(D) = \frac{1}{3} + \frac{1}{3} = \frac{2}{3}$.}
\end{exercise}

\index{event|)}
\index{disjoint|)}
\index{mutually exclusive|)}

\subsection{Probabilities when events are not disjoint}

Let's consider calculations for two events that are not disjoint in the context of a \indexthis{regular deck of 52 cards}{deck of cards}, represented in Table~\ref{deckOfCards}. If you are unfamiliar with the cards in a regular deck, please see the footnote.\footnote{The 52 cards are split into four \term{suits}: $\clubsuit$ (club), {\color{redcards}$\diamondsuit$} (diamond), {\color{redcards}$\heartsuit$} (heart), $\spadesuit$ (spade). Each suit has its 13 cards labeled: \resp{2}, \resp{3}, ..., \resp{10}, \resp{J} (jack), \resp{Q} (queen), \resp{K} (king), and \resp{A} (ace). Thus, each card is a unique combination of a suit and a label, e.g. {\color{redcards}\resp{4$\heartsuit$}} and \resp{J$\clubsuit$}. The 12 cards represented by the jacks, queens, and kings are called \termsub{\resp{face cards}}{face card}. The cards that are {\color{redcards}$\diamondsuit$} or {\color{redcards}$\heartsuit$} are typically colored {\color{redcards}red} while the other two suits are typically colored black.}

\begin{table}
\centering
\begin{tabular}{lll lll lll lll l}
\resp{2$\clubsuit$} & \resp{3$\clubsuit$} & \resp{4$\clubsuit$} & \resp{5$\clubsuit$} & \resp{6$\clubsuit$} & \resp{7$\clubsuit$} & \resp{8$\clubsuit$} & \resp{9$\clubsuit$} & \resp{10$\clubsuit$} & \resp{J$\clubsuit$} & \resp{Q$\clubsuit$} & \resp{K$\clubsuit$} & \resp{A$\clubsuit$}  \\
\color{redcards} \resp{2$\diamondsuit$} & \color{redcards}\resp{3$\diamondsuit$} & \color{redcards}\resp{4$\diamondsuit$} & \color{redcards}\resp{5$\diamondsuit$} & \color{redcards}\resp{6$\diamondsuit$} & \color{redcards}\resp{7$\diamondsuit$} & \color{redcards}\resp{8$\diamondsuit$} & \color{redcards}\resp{9$\diamondsuit$} & \color{redcards}\resp{10$\diamondsuit$} & \color{redcards}\resp{J$\diamondsuit$} & \color{redcards}\resp{Q$\diamondsuit$} & \color{redcards}\resp{K$\diamondsuit$} & \color{redcards}\resp{A$\diamondsuit$} \\
\color{redcards}\resp{2$\heartsuit$} & \color{redcards}\resp{3$\heartsuit$} & \color{redcards}\resp{4$\heartsuit$} & \color{redcards}\resp{5$\heartsuit$} & \color{redcards}\resp{6$\heartsuit$} & \color{redcards}\resp{7$\heartsuit$} & \color{redcards}\resp{8$\heartsuit$} & \color{redcards}\resp{9$\heartsuit$} & \color{redcards}\resp{10$\heartsuit$} & \color{redcards}\resp{J$\heartsuit$} & \color{redcards}\resp{Q$\heartsuit$} & \color{redcards}\resp{K$\heartsuit$} & \color{redcards}\resp{A$\heartsuit$} \\
\resp{2$\spadesuit$} & \resp{3$\spadesuit$} & \resp{4$\spadesuit$} & \resp{5$\spadesuit$} & \resp{6$\spadesuit$} & \resp{7$\spadesuit$} & \resp{8$\spadesuit$} & \resp{9$\spadesuit$} & \resp{10$\spadesuit$} & \resp{J$\spadesuit$} & \resp{Q$\spadesuit$} & \resp{K$\spadesuit$} & \resp{A$\spadesuit$}
\end{tabular}
\caption{Representations of the 52 unique cards in a deck.}
\label{deckOfCards}
\end{table}

\begin{exercise}
(a) What is the probability that a randomly selected card is a diamond? (b)~What is the probability that a randomly selected card is a face~card?\footnote{(a) There are 52 cards and 13 diamonds. If the cards are thoroughly shuffled, each card has an equal chance of being drawn, so the probability that a randomly selected card is a diamond is $P({\color{redcards}\diamondsuit}) = \frac{13}{52} = 0.250$. (b)~Likewise, there are 12 face cards, so $P($face card$) = \frac{12}{52} = \frac{3}{13} = 0.231$.}
\end{exercise}

\term{Venn diagrams} are useful when outcomes can be categorized as ``in'' or ``out'' for two or three variables, attributes, or random processes. The Venn diagram in Figure~\ref{venn} uses a circle to represent diamonds and another to represent face cards. If a card is both a diamond and a face card, it falls into the intersection of the circles. If it is a diamond but not a face card, it will be in part of the left circle that is not in the right circle (and so on). The total number of cards that are diamonds is given by the total number of cards in the diamonds circle: $10+3=13$. The probabilities are also shown (e.g. $10/52 = 0.1923$).

\begin{figure}
\centering
\includegraphics[height=1.4in]{ch_probability/figures/venn/venn}
\caption{A Venn diagram for diamonds and face cards.}
\label{venn}
\end{figure}

\begin{exercise}
Using the Venn diagram, verify $P($face card$) = 12/52=3/13$.\footnote{The Venn diagram shows face cards split up into ``face card but not {\color{redcards}$\diamondsuit$}'' and ``face card and {\color{redcards}$\diamondsuit$}''. Since these correspond to disjoint events, $P($face card$)$ is found by adding the two corresponding probabilities: $\frac{3}{52} + \frac{9}{52} = \frac{12}{52} = \frac{3}{13}$.}
\end{exercise}

Let $A$ represent the event that a randomly selected card is a diamond and $B$ represent the event that it is a face card. How do we compute $P(A$ or $B)$? Events $A$ and $B$ are not disjoint -- the cards {\color{redcards}$J\diamondsuit$}, {\color{redcards}$Q\diamondsuit$}, and {\color{redcards}$K\diamondsuit$} fall into both categories -- so we cannot use the Addition Rule for disjoint events. Instead we use the Venn diagram. We start by adding the probabilities of the two events:
\begin{eqnarray*}
P(A) + P(B) = P({\color{redcards}\diamondsuit}) + P(\text{face card}) = 13/52 + 12/52
\label{overCountFaceDiamond}
\end{eqnarray*}
However, the three cards that are in both events were counted twice, once in each probability. We must correct this double counting:
\begin{eqnarray}
P(A\text{ or } B) &=&P({\color{redcards}\diamondsuit}) + P(\text{face card})  \notag \\
 &=& P({\color{redcards}\diamondsuit}) + P(\text{face card}) - P({\color{redcards}\diamondsuit}  \text{ and face card}) \label{diamondFace} \\
 &=& 13/52 + 12/52 - 3/52 \notag \\
 &=& 22/52 = 11/26 \notag
\end{eqnarray}
Equation~(\ref{diamondFace}) is an example of the \term{General Addition Rule}.

\begin{termBox}{\tBoxTitle{General Addition Rule} If $A$ and $B$ are any two events, disjoint or not, then the probability that A or B will occur is
\begin{eqnarray}
P(A\text{ or }B) = P(A) + P(B) - P(A\text{ and }B)
\label{generalAdditionRule}
\end{eqnarray}
where $P(A$ and $B)$ is the probability that both events occur.}
\end{termBox}


\begin{tipBox}{\tipBoxTitle{Symbolic notation for ``and" and ``or"}
The symbol $\cap$ means intersection and is equivalent to ``and".

The symbol  $\cup$ means union and is equivalent to ``or".

It is common to see the General Addition Rule written as
\begin{eqnarray}
P(A \cup B) = P(A) + P(B) - P(A \cap B)
\end{eqnarray}
}
\end{tipBox}

\begin{tipBox}{\tipBoxTitle{``or'' is inclusive}
When we write, ``or"  in statistics, we mean ``and/or'' unless we explicitly state otherwise. Thus, $A$ or $B$ occurs means $A$, $B$, or both $A$ and $B$ occur. This is equivalent to at least one of $A$ or $B$ occurring.}
\end{tipBox}

\begin{exercise}
(a) If $A$ and $B$ are disjoint, describe why this implies $P(A$ and $B) = 0$. (b) Using part (a), verify that the General Addition Rule simplifies to the simpler Addition Rule for disjoint events if $A$ and $B$ are disjoint.\footnote{(a) If $A$ and $B$ are disjoint, $A$ and $B$ can never occur simultaneously. (b) If $A$ and $B$ are disjoint, then the last term of Equation~(\ref{generalAdditionRule}) is 0 (see part (a)) and we are left with the Addition Rule for disjoint events.}
\end{exercise}

\index{data!email|(}

\begin{exercise}\label{emailSpamNumberVennExer}
% library(openintro); data(email); table(email[,c("spam", "number")]); table(email[,c("number")]); table(email[,c("spam")])
In the \data{email} data set with 3,921 emails, 367 were spam, 2,827 contained some small numbers but no big numbers, and 168 had both characteristics. Create a Venn diagram for this setup.\footnote{%
\begin{minipage}[t]{0.65\textwidth}
Both the counts and corresponding {\color{oiB}probabilities} (e.g. $2659/3921 = 0.678$) are shown. Notice that the number of emails represented in the left circle corresponds to $2659 + 168 = 2827$, and the number represented in the right circle is $168 + 199 = 367$.
\end{minipage}\ %
\begin{minipage}[c]{0.3\textwidth}
\hfill\includegraphics[height=13mm]{ch_probability/figures/emailSpamNumberVenn/emailSpamNumberVenn} \vspace{-13mm}
\end{minipage}
}
\end{exercise}

\begin{exercise}
(a) Use your Venn diagram from Guided Practice~\ref{emailSpamNumberVennExer} to determine the probability a randomly drawn email from the \data{email} data set is spam and had small numbers (but not big numbers). (b)~What is the probability that the email had either of these attributes?\footnote{(a)~The solution is represented by the intersection of the two circles: 0.043. (b)~This is the sum of the three disjoint probabilities shown in the circles: $0.678 + 0.043 + 0.051 = 0.772$.}
\index{data!email|)}
\end{exercise}


\subsection{Complement of an event}

Rolling a die produces a value in the set $\{$\resp{1}, \resp{2}, \resp{3}, \resp{4}, \resp{5}, \resp{6}$\}$. This set of all possible outcomes is called the \term{sample space} ($S$)\marginpar[\raggedright\vspace{-5mm}

$S$\\\footnotesize Sample space]{\raggedright\vspace{-5mm}

$S$\\\footnotesize Sample space}\index{S@$S$} for rolling a die. We often use the sample space to examine the scenario where an event does not occur.

Let $D=\{$\resp{2}, \resp{3}$\}$ represent the event that the outcome of a die roll is \resp{2} or \resp{3}. Then the \term{complement}\marginpar[\raggedright\vspace{0.2mm}

$A^c$\\\footnotesize Complement\\of outcome $A$]{\raggedright\vspace{0.2mm}

$A^c$\\\footnotesize Complement\\of outcome $A$}\index{Ac@$A^c$} of $D$ represents all outcomes in our sample space that are not in $D$, which is denoted by $D^c = \{$\resp{1}, \resp{4}, \resp{5}, \resp{6}$\}$. That is, $D^c$ is the set of all possible outcomes not already included in $D$. Figure~\ref{complementOfD} shows the relationship between $D$, $D^c$, and the sample space $S$.

\begin{figure}[hht]
\centering
\includegraphics[width=0.5\textwidth]{ch_probability/figures/complementOfD/complementOfD}
\caption{Event $D=\{$\resp{2}, \resp{3}$\}$ and its complement, $D^c = \{$\resp{1}, \resp{4}, \resp{5}, \resp{6}$\}$. $S$~represents the sample space, which is the set of all possible events.}
\label{complementOfD}

\end{figure}

\begin{exercise}
(a) Compute $P(D^c) = P($rolling a \resp{1}, \resp{4}, \resp{5}, or \resp{6}$)$. (b) What is $P(D) + P(D^c)$?\footnote{(a)~The outcomes are disjoint and each has probability $1/6$, so the total probability is $4/6=2/3$. (b)~We can also see that $P(D)=\frac{1}{6} + \frac{1}{6} = 1/3$. Since $D$ and $D^c$ are disjoint, $P(D) + P(D^c) = 1$.}
\end{exercise}

\begin{exercise}
Events $A=\{$\resp{1}, \resp{2}$\}$ and $B=\{$\resp{4}, \resp{6}$\}$ are shown in Figure~\ref{disjointSets} on page~\pageref{disjointSets}. (a) Write out what $A^c$ and $B^c$ represent. (b)~Compute $P(A^c)$ and $P(B^c)$. (c)~Compute $P(A)+P(A^c)$ and $P(B)+P(B^c)$.\footnote{Brief solutions: (a)~$A^c=\{$\resp{3}, \resp{4}, \resp{5}, \resp{6}$\}$ and $B^c=\{$\resp{1}, \resp{2}, \resp{3}, \resp{5}$\}$. (b)~Noting that each outcome is disjoint, add the individual outcome probabilities to get $P(A^c)=2/3$ and $P(B^c)=2/3$. (c)~$A$~and~$A^c$ are disjoint, and the same is true of $B$~and~$B^c$. Therefore, $P(A) + P(A^c) = 1$ and $P(B) + P(B^c) = 1$.}
\end{exercise}

An event $A$ together with its complement $A^c$ comprise the entire sample space. Because of this we can say that $P(A) + P(A^c) = 1$.

\begin{termBox}{\tBoxTitle{Complement}
The complement of event $A$ is denoted $A^c$, and $A^c$ represents all outcomes not in~$A$. $A$ and $A^c$ are mathematically related: \vspace{-2mm}
\begin{eqnarray}\label{complement}
P(A) + P(A^c) = 1, \quad\text{i.e.}\quad P(A) = 1-P(A^c)
\end{eqnarray}\vspace{-6.5mm}}
\end{termBox}

In simple examples, computing $A$ or $A^c$ is feasible in a few steps. However, using the complement can save a lot of time as problems grow in complexity.

\begin{exercise}
A die is rolled 10 times. (a)~What is the complement of getting at least one 6 in 10 rolls of the die? (b)~What is the complement of getting at most three 6's in 10 rolls of the die?\footnote{(a)~The complement of getting at least one 6 in ten rolls of a die is getting zero 6's in the 10 rolls. (b)~The complement of getting at most three 6's in 10 rolls is getting four, five, ..., nine, or ten 6's in 10~rolls.}
\end{exercise}

%\Cut{
%\begin{exercise}
%Let $A$ represent the event where we roll two dice and their total is less than \resp{12}. (a) What does the event $A^c$ represent? (b) Determine $P(A^c)$ from Table~\ref{diceProb} on page~\pageref{diceProb}. (c) Determine $P(A)$.\footnote{(a)~The complement of $A$: when the total is equal to \resp{12}. (b)~$P(A^c) = 1/36$. (c)~Use the probability of the complement from part (b), $P(A^c) = 1/36$, and Equation~(\ref{complement}): $P($less than \resp{12}$) = 1 - P($\resp{12}$) = 1 - 1/36 = 35/36$.}
%\end{exercise}
%
%\begin{exercise} Consider again the probabilities from Table~\ref{diceProb} and rolling two dice. Find the following probabilities: (a)~The sum of the dice is \emph{not} \resp{6}. (b)~The sum is at least \resp{4}. That is, determine the probability of the event $B=\{$\resp{4}, \resp{5}, ..., \resp{12}$\}$. (c) The sum is no more than \resp{10}. That is, determine the probability of the event $D=\{$\resp{2}, \resp{3}, ..., \resp{10}$\}$.\footnote{(a)~First find $P($\resp{6}$)=5/36$, then use the complement: $P($not \resp{6}$) = 1 - P($\resp{6}$) = 31/36$.
%
%(b)~First find  the complement, which requires much less effort: $P($\resp{2} or \resp{3}$)=1/36+2/36=1/12$. Then calculate $P(B) = 1-P(B^c) = 1-1/12 = 11/12$.
%
%(c)~As before, finding the complement is the clever way to determine $P(D)$. First find $P(D^c) = P($\resp{11} or \resp{12}$)=2/36 + 1/36=1/12$. Then calculate $P(D) = 1 - P(D^c) = 11/12$.}
%\end{exercise}
%}


\subsection{Independence}
\label{probabilityIndependence}

Just as variables and observations can be independent, random processes can be independent, too. Two processes are \term{independent} if knowing the outcome of one provides no useful information about the outcome of the other. For instance, flipping a coin and rolling a die are two independent processes -- knowing the coin was heads does not help determine the outcome of a die roll. On the other hand, stock prices usually move up or down together, so they are not independent.

Example~\ref{probOf2Ones} provides a basic example of two independent processes: rolling two dice. We want to determine the probability that both will be \resp{1}. Suppose one of the dice is red and the other white. If the outcome of the red die is a \resp{1}, it provides no information about the outcome of the white die. We first encountered this same question in Example~\ref{probOf2Ones} (page~\pageref{probOf2Ones}), where we calculated the probability using the following reasoning: $1/6^{th}$ of the time the red die is a \resp{1}, and $1/6^{th}$ of \emph{those} times the white die will also be \resp{1}. This is illustrated in Figure~\ref{indepForRollingTwo1s}. Because the rolls are independent, the probabilities of the corresponding outcomes can be multiplied to get the final answer: $(1/6)\times(1/6)=1/36$. This can be generalized to many independent processes.

\begin{figure}[hht]
\centering
\includegraphics[width=0.7\textwidth]{ch_probability/figures/indepForRollingTwo1s/indepForRollingTwo1s}
\caption{$1/6^{th}$ of the time, the first roll is a \resp{1}. Then $1/6^{th}$ of \emph{those} times, the second roll will also be a \resp{1}.}
\label{indepForRollingTwo1s}
\end{figure}

\begin{example}{What if there was also a blue die independent of the other two? What is the probability of rolling the three dice and getting all \resp{1}s?}\label{threeDice}
The same logic applies from Example~\ref{probOf2Ones}. If $1/36^{th}$ of the time the white and red dice are both \resp{1}, then $1/6^{th}$ of \emph{those} times the blue die will also be \resp{1}, so multiply:
{\begin{align*}
P(white=\text{\small\resp{1} and } red=\text{\small\resp{1} and } blue=\text{\small\resp{1}})
	&= P(white=\text{\small\resp{1}})\times P(red=\text{\small\resp{1}})\times P(blue=\text{\small\resp{1}}) \\
	&= (1/6)\times (1/6)\times (1/6)
	= 1/216
\end{align*}} \vspace{-7mm}
\end{example}

Examples~\ref{probOf2Ones} and~\ref{threeDice} illustrate what is called the Multiplication Rule for independent processes.

\begin{termBox}{\tBoxTitle{\term{Multiplication Rule} for independent processes}
If $A$ and $B$ represent events from two different and independent processes, then the probability that both $A$ and $B$ occur can be calculated as the product of their separate probabilities: \vspace{-1.5mm}
\begin{eqnarray}\label{eqForIndependentEvents}
P(A \text{ and }B) = P(A) \times  P(B)
\end{eqnarray}
Similarly, if there are $k$ events $A_1$, ..., $A_k$ from $k$ independent processes, then the probability they all occur is\vspace{-1.5mm}
\begin{eqnarray*}
P(A_1) \times  P(A_2)\times  \cdots \times  P(A_k)
\end{eqnarray*}\vspace{-6mm}}
\end{termBox}

\begin{exercise} \label{ex2Handedness}
About 9\% of people are left-handed. Suppose 2 people are selected at random from the U.S. population. Because the sample size of 2 is very small relative to the population, it is reasonable to assume these two people are independent. (a)~What is the probability that both are left-handed? (b)~What is the probability that both are right-handed?\footnote{(a) The probability the first person is left-handed is $0.09$, which is the same for the second person. We apply the Multiplication Rule for independent processes to determine the probability that both will be left-handed: $0.09\times 0.09 = 0.0081$.

(b) It is reasonable to assume the proportion of people who are ambidextrous (both right and left handed) is nearly 0, which results in $P($right-handed$)=1-0.09=0.91$. Using the same reasoning as in part~(a), the probability that both will be right-handed is $0.91\times 0.91 = 0.8281$.}
\end{exercise}

\begin{exercise} \label{ex5Handedness}
Suppose 5 people are selected at random.\footnote{(a)~The abbreviations \resp{RH} and \resp{LH} are used for right-handed and left-handed, respectively. Since each are independent, we apply the Multiplication Rule for independent processes:
\begin{align*}
P(\text{all five are \resp{RH}})
&= P(\text{first = \resp{RH}, second = \resp{RH}, ..., fifth = \resp{RH}}) \\
&= P(\text{first = \resp{RH}})\times P(\text{second = \resp{RH}})\times  \dots \times P(\text{fifth = \resp{RH}}) \\
&= 0.91\times 0.91\times 0.91\times 0.91\times 0.91 = 0.624
\end{align*}

(b)~Using the same reasoning as in~(a), $0.09\times 0.09\times 0.09\times 0.09\times 0.09 = 0.0000059$

(c)~Use the complement, $P($all five are \resp{RH}$)$, to answer this question:
\begin{align*}
P(\text{not all \resp{RH}})
	= 1 - P(\text{all \resp{RH}})
	= 1 - 0.624 = 0.376
\end{align*}} \vspace{-1.5mm}
\begin{enumerate}
\setlength{\itemsep}{0mm}
\item[(a)] What is the probability that all are right-handed?
\item[(b)] What is the probability that all are left-handed?
\item[(c)] What is the probability that not all of the people are right-handed?
\end{enumerate}
\end{exercise}

Suppose the variables \var{handedness} and \var{gender} are independent, i.e. knowing someone's \var{gender} provides no useful information about their \var{handedness} and vice-versa. Then we can compute whether a randomly selected person is right-handed and female\footnote{The actual proportion of the U.S. population that is \resp{female} is about 50\%, and so we use 0.5 for the probability of sampling a woman. However, this probability does differ in other countries.} using the Multiplication Rule:
\begin{eqnarray*}
P(\text{right-handed and female}) &=& P(\text{right-handed}) \times  P(\text{female}) \\
&=& 0.91 \times  0.50 = 0.455
\end{eqnarray*}


\begin{exercise}
Three people are selected at random.\footnote{Brief answers are provided. (a)~This can be written in probability notation as $P($a randomly selected person is male and right-handed$)=0.455$. (b) 0.207. (c) 0.045. (d) 0.0093.} \vspace{-1.5mm}
\begin{enumerate}
\setlength{\itemsep}{0mm}
\item[(a)] What is the probability that the first person is male and right-handed?
\item[(b)] What is the probability that the first two people are male and right-handed?.
\item[(c)] What is the probability that the third person is female and left-handed?
\item[(d)] What is the probability that the first two people are male and right-handed and the third person is female and left-handed?
\end{enumerate}
\end{exercise}

Sometimes we wonder if one outcome provides useful information about another outcome. The question we are asking is, are the occurrences of the two events independent? We say that two events $A$ and $B$ are independent if they satisfy Equation~\eqref{eqForIndependentEvents}.

\begin{example}{If we shuffle up a deck of cards and draw one, is the event that the card is a heart independent of the event that the card is an ace?}
The probability the card is a heart is $1/4$ and the probability that it is an ace is $1/13$. The probability the card is the ace of hearts is $1/52$. We check whether Equation~\ref{eqForIndependentEvents} is satisfied:
\begin{align*}
P({\color{redcards}\heartsuit})\times P(\text{ace}) = \frac{1}{4}\times \frac{1}{13} = \frac{1}{52}
					= P({\color{redcards}\heartsuit}\text{ and ace})
\end{align*}
Because the equation holds, the event that the card is a heart and the event that the card is an ace are independent events.
\end{example}

%_________________
\section{Conditional probability}
\label{conditionalProbabilitySection}

\index{data!drug\_use|(}

Are students more likely to use marijuana when their parents used drugs? The \data{drug\_\hspace{0.3mm}use} data set contains a sample of 445 cases with two variables, \var{student} and \var{parents}, and is summarized in Table~\ref{contTableOfParStDrugUse}.\footnote{Ellis GJ and Stone LH. 1979. Marijuana Use in College: An Evaluation of a Modeling Explanation. Youth and Society 10:323-334.} The \var{student} variable is either \resp{uses} or \resp{not}, where a student is labeled as \var{uses} if she has recently used marijuana. The \var{parents} variable takes the value \resp{used} if at least one of the parents used drugs, including alcohol.

\begin{table}[ht]
\centering
\begin{tabular}{ll rr r rr}
  && \multicolumn{2}{c}{\var{parents}} & \hspace{1cm} &  \\
  \cline{3-4}
	&& \resp{used} & \resp{not} & Total  \\
  \cline{2-5}
	& \resp{uses}     & 125 & 94 & 219 \\
\raisebox{1.5ex}[0pt]{\var{student}}	& \resp{not} \hspace{0.5cm} & 85 & 141 & 226   \\
  \cline{2-5}
	& Total & 210 & 235 & 445 \\
\end{tabular}
\caption{Contingency table summarizing the \data{drug\_\hspace{0.3mm}use} data set.}
\label{contTableOfParStDrugUse}
\end{table}

\begin{figure}[ht]
\centering
\includegraphics[width=0.65\textwidth]{ch_probability/figures/drugUseVenn/drugUseVenn}
\caption{A Venn diagram using boxes for the \data{drug\_\hspace{0.3mm}use} data set.}
\label{drugUseVenn}
\end{figure}

\begin{example}{If at least one parent used drugs, what is the chance their child (\var{student}) {uses}?}
We will estimate this probability using the data. Of the 210 cases in this data set where \var{parents} \resp{used}, 125 represent cases where \var{student} \resp{uses}:
\begin{eqnarray*}
P(\text{\var{student} \resp{uses} given \var{parents} \resp{used}}) = \frac{125}{210} = 0.60
\end{eqnarray*}
\end{example}

\begin{example}{A student is randomly selected from the study and she does not use drugs. What is the probability that at least one of her parents used?}\label{drugUseProbOfParentsGivenStudents}
If the student does not use drugs, then she is one of the 226 students in the second row. Of these 226 students, 85 had at least one parent who used drugs:
\begin{eqnarray*}
P(\text{\var{parents} \resp{used} given \var{student} \resp{does not use}}) = \frac{85}{226} = 0.376
\end{eqnarray*}
\end{example}

\subsection{Marginal and joint probabilities}
\label{marginalAndJointProbabilities}

\index{marginal probability|(}
\index{joint probability|(}

Table~\ref{drugUseProbTable} includes row and column totals for each variable separately in the \data{drug\_\hspace{0.3mm}use} data set. These totals represent \termsub{marginal probabilities}{marginal probability} for the sample, which are the probabilities based on a single variable without conditioning on any other variables. For instance, a probability based solely on the \var{student} variable is a marginal probability:
\begin{eqnarray*}
P(\text{\var{student} \resp{uses}}) = \frac{219}{445} = 0.492
\end{eqnarray*}
A probability of outcomes for two or more variables or processes is called a \term{joint probability}:
\begin{eqnarray*}
P(\text{\var{student} \resp{uses} and \var{parents} \resp{did not use}}) = \frac{94}{445} = 0.21
\end{eqnarray*}
It is common to substitute a comma for ``and'' in a joint probability, although either is acceptable.

\begin{table}
\centering
\begin{tabular}{l rr r}
  \hline
& \var{parents}: \resp{used} & \var{parents}: \resp{not} & Total  \\
  \hline
\var{student}: \resp{uses}     & 0.28 & 0.21 & 0.49 \\
\var{student}: \resp{not} \hspace{0.5cm} & 0.19 & 0.32 & 0.51  \\
   \hline
Total & 0.47 & 0.53 & 1.00 \\
\hline
\end{tabular}
\caption{Probability table summarizing parental and student drug use.}
\label{drugUseProbTable}
\end{table}

\begin{termBox}{\tBoxTitle{Marginal and joint probabilities}
If a probability is based on a single variable, it is a \emph{marginal probability}. The probability of outcomes for two or more variables or processes is called a \emph{joint probability}.}
\end{termBox}

We use \term{table proportions} to summarize joint probabilities for the \data{drug\_\hspace{0.3mm}use} sample. These proportions are computed by dividing each count in Table~\ref{contTableOfParStDrugUse} by 445 to obtain the proportions in Table~\ref{drugUseProbTable}. The joint probability distribution of the \var{parents} and \var{student} variables is shown in Table~\ref{drugUseDistribution}.

\begin{table}
\centering
\begin{tabular}{l c}
  \hline
Joint outcome & Probability \\
  \hline
\var{parents} \resp{used}, \var{student} \resp{uses} & 0.28 \\
\var{parents} \resp{used}, \var{student} \resp{does not use} & 0.19 \\
\var{parents} \resp{did not use}, \var{student} \resp{uses} & 0.21 \\
\var{parents} \resp{did not use}, \var{student} \resp{does not use} & 0.32 \\
   \hline
Total & 1.00 \\
\hline
\end{tabular}
\caption{A joint probability distribution for the \data{drug\_\hspace{0.3mm}use} data set.}
\label{drugUseDistribution}
\end{table}

\begin{exercise}
Verify Table~\ref{drugUseDistribution} represents a probability distribution: events are disjoint, all probabilities are non-negative, and the probabilities sum to~1.\footnote{Each of the four outcome combination are disjoint, all probabilities are indeed non-negative, and the sum of the probabilities is $0.28 + 0.19 + 0.21 + 0.32 = 1.00$.}
\end{exercise}

We can compute marginal probabilities using joint probabilities in simple cases. For example, the probability a random student from the study uses drugs is found by summing the outcomes from Table~\ref{drugUseDistribution} where \var{student} \resp{uses}:\index{marginal probability|)}\index{joint probability|)}
\begin{align*}
P(\text{\underline{\color{black}\var{student} \resp{uses}}})
&=  P(\text{\var{parents} \resp{used}, \underline{\color{black}\var{student} \resp{uses}}}) \\
& \quad \quad + P(\text{\var{parents} \resp{did not use}, \underline{\color{black}\var{student} \resp{uses}}}) \\
&= 0.28 + 0.21 \\
&= 0.49
\end{align*}


\subsection{Defining conditional probability}

\index{conditional probability|(}

There is some connection between drug use of parents and of the student: drug use of one is associated with drug use of the other.\footnote{This is an observational study and no causal conclusions may be reached.} In this section, we discuss how to use information about associations between two variables to improve probability estimation.

The probability that a random student from the study uses drugs is 0.49. Could we update this probability if we knew that this student's parents used drugs? Absolutely. To do so, we limit our view to only those 210 cases where parents used drugs and look at the fraction where the student uses drugs:\vspaceB{-1mm}
\begin{eqnarray*}
P(\text{\var{student} \resp{uses} given \var{parents} \resp{used}}) = \frac{125}{210} = 0.60
\end{eqnarray*}
We call this a \term{conditional probability} because we computed the probability under a condition: \var{parents} \resp{used}. There are two parts to a conditional probability, \term{the outcome of interest} and the \term{condition}. It is useful to think of the condition as information we know to be true, and this information usually can be described as a known outcome or~event.

We separate the text inside our probability notation into the outcome of interest and the condition:\vspaceB{-2.5mm}
\begin{eqnarray}
&& P(\text{\var{student} \resp{uses} given \var{parents} \resp{used}}) \notag \\
&& = P(\text{\var{student} \resp{uses}} | \text{ \var{parents} \resp{used}}) = \frac{125}{210} = 0.60
\label{probStudentUsedIfParentsUsedInFormalNotation}
\end{eqnarray}
\marginpar[\raggedright\vspace{-10mm}

$P(A | B)$\vspace{1mm}\\\footnotesize Probability of\\outcome $A$\\given $B$]{\raggedright\vspace{-10mm}

$P(A | B)$\vspace{1mm}\\\footnotesize Probability of\\outcome $A$\\given $B$}The vertical bar ``$|$'' is read as \emph{given}.

In Equation~(\ref{probStudentUsedIfParentsUsedInFormalNotation}), we computed the probability a student uses based on the condition that at least one parent used as a fraction:\vspaceB{-1mm}
\begin{eqnarray}
&& P(\text{\var{student} \resp{uses}} | \text{ \var{parents} \resp{used}}) \notag \\
&&\quad = \frac{\text{\# times \var{student} \resp{uses} and \var{parents} \resp{used}}}{\text{\# times \var{parents} \resp{used}}} \label{ratioOfBothToRatioOfConditionalForParentsAndStudent} \\
&&\quad = \frac{125}{210} = 0.60 \notag
\end{eqnarray}
We considered only those cases that met the condition, \var{parents} \resp{used}, and then we computed the ratio of those cases that satisfied our outcome of interest, the student~uses.

Counts are not always available for data, and instead only marginal and joint probabilities may be provided. For example, disease rates are commonly listed in percentages rather than in a count format. We would like to be able to compute conditional probabilities even when no counts are available, and we use Equation~(\ref{ratioOfBothToRatioOfConditionalForParentsAndStudent}) as an example demonstrating this technique.

We considered only those cases that satisfied the condition, \var{parents} \resp{used}. Of these cases, the conditional probability was the fraction who represented the outcome of interest, \var{student} \resp{uses}. Suppose we were provided only the information in Table~\vref{drugUseProbTable}, i.e. only probability data. Then if we took a sample of 1000 people, we would anticipate about 47\% or $0.47\times 1000 = 470$ would meet our information criterion. Similarly, we would expect about 28\% or $0.28\times 1000 = 280$ to meet both the information criterion and represent our outcome of interest. Thus, the conditional probability could be computed:\vspaceB{-1mm}
\begin{align}
P(\text{\var{student} \resp{uses}} | \text{ \var{parents} \resp{used}})
	&= \frac{\text{\# (\var{student} \resp{uses} and \var{parents} \resp{used})}}{\text{\# (\var{parents} \resp{used})}} \notag \\
	&= \frac{280}{470} = \frac{0.28}{0.47} = 0.60
\label{stUserPUsedHypSampSize}
\end{align}
In Equation~(\ref{stUserPUsedHypSampSize}), we examine exactly the fraction of two probabilities, 0.28 and 0.47, which we can write as
\begin{align*}
P(\text{\var{student} \resp{uses} and \var{parents} \resp{used}})
	\quad\text{and}\quad
	P(\text{\var{parents} \resp{used}}).
\end{align*}
The fraction of these probabilities represents our general formula for conditional probability.

\begin{termBox}{\tBoxTitle{Conditional Probability}
The conditional probability of the outcome of interest $A$ given condition $B$ is computed as the following: \vspace{-1.5mm}
\begin{eqnarray}
P(A | B) = \frac{P(A\text{ and }B)}{P(B)}
\label{condProbEq}
\end{eqnarray}}
\end{termBox}

\begin{exercise}\label{drugUseProbOfParentsEqualNotGivenStudents}
(a) Write out the following statement in conditional probability notation: ``\emph{The probability a random case has \var{parents} \resp{did not use} if it is known that \var{student} \resp{does not use}}''. Notice that the condition is now based on the {student}, not the {parent}. (b)~Determine the probability from part (a). Table~\vref{drugUseProbTable} may be helpful.\footnote{(a) $P(\text{\var{parents}  \resp{did not use}} | \text{ \var{student} \resp{does not use}})$. (b) Equation~(\ref{condProbEq}) for conditional probability indicates we should first find $P(\text{\var{parents} \resp{did not use} and \var{student} \resp{does not use}})=0.32$ and $P(\text{\var{student} \resp{does not use}})=0.51$. Then the ratio represents the conditional probability: $0.32/0.51 = 0.63$.}
\end{exercise}

\begin{exercise}\label{whyCondProbSumTo1}
(a)~Determine the probability that one of the parents had used drugs if it is known the student does not use drugs.
(b)~Using the answers from part (a) and Guided Practice~\ref{drugUseProbOfParentsEqualNotGivenStudents}(b), compute \vspace{-1.5mm}
\begin{eqnarray*}
&&P(\text{\var{parents} \resp{used}}|\text{\var{student} \resp{does not use}})\\
&&	\quad \quad \quad \quad + P(\text{\var{parents} \resp{did not use}}|\text{\var{student} \resp{does not use}})
\end{eqnarray*}
(c)~Provide an intuitive argument to explain why the sum in (b) is 1.\footnote{(a)~This probability is $\frac{P(\text{\var{parents} \resp{used} and \var{student} \resp{does not use}})}{P(\text{\var{student} \resp{does not use}})} = \frac{0.19}{0.51} = 0.37$. (b)~The total equals~1. (c)~Under the condition the student does not use drugs, the parents must either use drugs or not. The complement still appears to work \emph{when conditioning on the same information}.}
\end{exercise}

\begin{exercise}
The data indicate that drug use of parents and children are associated. Does this mean the drug use of parents causes the drug use of the students?\footnote{No. This was an observational study. Two potential confounding variables include \var{income} and \var{region}. Can you think of others?}
\index{conditional probability|)}
\index{data!drug\_use|)}
\end{exercise}

\subsection{Smallpox in Boston, 1721}

\index{data!smallpox|(}

The \data{smallpox} data set provides a sample of 6,224 individuals from the year 1721 who were exposed to smallpox in Boston.\footnote{Fenner F. 1988. \emph{Smallpox and Its Eradication (History of International Public Health, No. 6)}. Geneva: World Health Organization. ISBN 92-4-156110-6.} Doctors at the time believed that inoculation, which involves exposing a person to the disease in a controlled form, could reduce the likelihood of death.

Each case represents one person with two variables: \var{inoculated} and \var{result}. The variable \var{inoculated} takes two levels: \resp{yes} or \resp{no}, indicating whether the person was inoculated or not. The variable \var{result} has outcomes \resp{lived} or \resp{died}. These data are summarized in Tables~\ref{smallpoxContingencyTable} and~\ref{smallpoxProbabilityTable}.

\begin{table}
\centering
\begin{tabular}{ll rr r}
& & \multicolumn{2}{c}{inoculated} & \\
\cline{3-4}
& & \resp{yes} & \resp{no} & Total  \\
\cline{2-5}
		& \resp{lived}     & 238 & 5136 & 5374 \\
\raisebox{1.5ex}[0pt]{\var{result}} &  \resp{died} \hspace{0.5cm} & 6 & 844 & 850  \\
\cline{2-5}
	& Total & 244 & 5980 & 6224 \\
\end{tabular}
\caption{Contingency table for the \data{smallpox} data set.}
\label{smallpoxContingencyTable}
\end{table}

\begin{table}
\centering
\begin{tabular}{ll rr r}
& & \multicolumn{2}{c}{inoculated} & \\
\cline{3-4}
& & \resp{yes} & \resp{no} & Total  \\
   \cline{2-5}
 & \resp{lived}     & 0.0382 & 0.8252 & 0.8634 \\
\raisebox{1.5ex}[0pt]{\var{result}} & \resp{died} \hspace{0.5cm} & 0.0010 & 0.1356  & 0.1366  \\
   \cline{2-5}
& Total & 0.0392 & 0.9608 & 1.0000 \\
\end{tabular}
\caption{Table proportions for the \data{smallpox} data, computed by dividing each count by the table total, 6224.}
\label{smallpoxProbabilityTable}
\end{table}

\begin{exercise} \label{probDiedIfNotInoculated}
Write out, in formal notation, the probability a randomly selected person who was not inoculated died from smallpox, and find this \mbox{probability.}\footnote{$P($\var{result} = \resp{died} $|$ \resp{not inoculated}$) = \frac{P(\text{\var{result} = \resp{died} and \resp{not inoculated}})}{P(\text{\resp{not inoculated}})} = \frac{0.1356}{0.9608} = 0.1411$.}
\end{exercise}

\begin{exercise}
Determine the probability that an inoculated person died from smallpox. How does this result compare with the result of Guided Practice~\ref{probDiedIfNotInoculated}?\footnote{$P($\resp{died} $|$ \resp{inoculated}$) = \frac{P(\text{\resp{died} and \resp{inoculated}})}{P(\text{\resp{inoculated}})} = \frac{0.0010}{0.0392} = 0.0255$. The death rate for individuals who were inoculated is only about 1~in~40 while the death rate is about 1~in~7 for those who were not inoculated.}
\end{exercise}

\begin{exercise}\label{SmallpoxInoculationObsExpExercise}
The people of Boston self-selected whether or not to be inoculated. (a) Is this study observational or was this an experiment? (b) Can we infer any causal connection using these data? (c) What are some potential confounding variables that might influence whether someone \resp{lived} or \resp{died} and also affect whether that person was inoculated?\footnote{Brief answers: (a)~Observational. (b)~No, we cannot infer causation from this observational study. (c)~Accessibility to the latest and best medical care. There are other valid answers for part~(c).}
\end{exercise}


\subsection{General multiplication rule}

Section~\ref{probabilityIndependence} introduced the Multiplication Rule for independent processes. Here we provide the \term{General Multiplication Rule} for events that might not be independent.

\begin{termBox}{\tBoxTitle{General Multiplication Rule}
If $A$ and $B$ represent two outcomes or events, then \vspace{-1.5mm}
\begin{eqnarray*}
P(A\text{ and }B) = P(A | B)\times P(B)
\end{eqnarray*} \vspace{-6.5mm} \par
For the term $P(A | B)$, it is useful to think of $A$ as the outcome of interest and $B$ as the condition.}
\end{termBox}
This General Multiplication Rule is simply a rearrangement of the definition for conditional probability in Equation~(\ref{condProbEq}) on page~\pageref{condProbEq}.

\begin{example}{Consider the \data{smallpox} data set. Suppose we are given only two pieces of information: 96.08\% of residents were not inoculated, and 85.88\% of the residents who were not inoculated ended up surviving. How could we compute the probability that a resident was not inoculated and lived?}
We will compute our answer using the General Multiplication Rule and then verify it using Table~\ref{smallpoxProbabilityTable}. We want to determine
\begin{eqnarray*}
P(\text{\resp{lived} and \resp{not inoculated}})
\end{eqnarray*}
and we are given that
\begin{eqnarray*}
P(\text{\resp{lived}}|\text{ \resp{not inoculated}})=0.8588 \\
P(\text{\resp{not inoculated}})=0.9608
\end{eqnarray*}
Among the 96.08\% of people who were not inoculated, 85.88\% survived:
\begin{eqnarray*}
P(\text{\resp{lived} and \resp{not inoculated}}) = 0.8588\times 0.9608 = 0.8251
\end{eqnarray*}
This is equivalent to the General Multiplication Rule. We can confirm this probability in Table~\ref{smallpoxProbabilityTable} at the intersection of \resp{no} and \resp{lived} (with a small rounding error).
\end{example}

\begin{exercise}
Use $P($\resp{inoculated}$) = 0.0392$ and $P($\resp{lived} $|$ \resp{inoculated}$) = 0.9754$ to determine the probability that a person was both inoculated and lived.\footnote{The answer is 0.0382, which can be verified using Table~\ref{smallpoxProbabilityTable}.}
\end{exercise}

\begin{exercise}
If 97.45\% of the people who were inoculated lived, what proportion of inoculated people must have died?\footnote{There were only two possible outcomes: \resp{lived} or \resp{died}. This means that 100\% - 97.45\% = 2.55\% of the people who were inoculated died.}
\end{exercise}

\begin{exercise}
Based on the probabilities computed above, does it appear that inoculation is effective at reducing the risk of death from smallpox?\footnote{The samples are large relative to the difference in death rates for the ``inoculated'' and ``not inoculated'' groups, so it seems there is an association between \var{inoculated} and \var{outcome}. However, as noted in the solution to Guided Practice~\ref{SmallpoxInoculationObsExpExercise}, this is an observational study and we cannot be sure if there is a causal connection. (Further research has shown that inoculation is effective at reducing death rates.)}
\end{exercise}


\subsection{Sampling from a small population}
\label{smallPop}

\begin{example}{Professors sometimes select a student at random to answer a question. If each student has an equal chance of being selected and there are 15 people in your class, what is the chance that she will pick you for the next question?}
If there are 15 people to ask and none are skipping class, then the probability is $1/15$, or about $0.067$.
\end{example}

\begin{example}{If the professor asks 3 questions, what is the probability that you will not be selected? Assume that she will not pick the same person twice in a given lecture.}\label{3woRep}
For the first question, she will pick someone else with probability $14/15$. When she asks the second question, she only has 14 people who have not yet been asked. Thus, if you were not picked on the first question, the probability you are again not picked is $13/14$. Similarly, the probability you are again not picked on the third question is $12/13$, and the probability of not being picked for any of the three questions is
\begin{eqnarray*}
&&P(\text{not picked in 3 questions}) \\
&&\quad = P(\text{\var{Q1}} = \text{\resp{not\_\hspace{0.3mm}picked}, }\text{\var{Q2}} = \text{\resp{not\_\hspace{0.3mm}picked}, }\text{\var{Q3}} = \text{\resp{not\_\hspace{0.3mm}picked}.}) \\
&&\quad = \frac{14}{15}\times\frac{13}{14}\times\frac{12}{13} = \frac{12}{15} = 0.80
\end{eqnarray*}
\end{example}

\begin{exercise}
What rule permitted us to multiply the probabilities in Example~\ref{3woRep}?\footnote{The three probabilities we computed were actually one marginal probability, $P($\var{Q1}$ = $\resp{not\_\hspace{0.3mm}picked}$)$, and two conditional probabilities:
\begin{eqnarray*}
&&P(\text{\var{Q2}} =  \text{\resp{not\_\hspace{0.3mm}picked}}|\text{ \var{Q1}} = \text{\resp{not\_\hspace{0.3mm}picked}}) \\
&&P(\text{\var{Q3}} =  \text{\resp{not\_\hspace{0.3mm}picked}}|\text{ \var{Q1}} = \text{\resp{not\_\hspace{0.3mm}picked}, }\text{\var{Q2}} = \text{\resp{not\_\hspace{0.3mm}picked}})
\end{eqnarray*}
Using the General Multiplication Rule, the product of these three probabilities is the probability of not being picked in 3 questions.}
\end{exercise}

\begin{example}{Suppose the professor randomly picks without regard to who she already selected, i.e. students can be picked more than once. What is the probability that you will not be picked for any of the three questions?}\label{3wRep}
Each pick is independent, and the probability of not being picked for any individual question is $14/15$. Thus, we can use the Multiplication Rule for independent processes.
\begin{eqnarray*}
&&P(\text{not picked in 3 questions}) \\
&&\quad = P(\text{\var{Q1}} = \text{\resp{not\_\hspace{0.3mm}picked}, }\text{\var{Q2}} = \text{\resp{not\_\hspace{0.3mm}picked}, }\text{\var{Q3}} = \text{\resp{not\_\hspace{0.3mm}picked}.}) \\
&&\quad = \frac{14}{15}\times\frac{14}{15}\times\frac{14}{15} = 0.813
\end{eqnarray*}
You have a slightly higher chance of not being picked compared to when she picked a new person for each question. However, you now may be picked more than once.
\end{example}

\begin{exercise}
Under the setup of Example~\ref{3wRep}, what is the probability of being picked to answer all three questions?\footnote{$P($being picked to answer all three questions$) = \left(\frac{1}{15}\right)^3 = 0.00030$.}
\end{exercise}

If we sample from a small population \term{without replacement}, we no longer have independence between our observations. In Example~\ref{3woRep}, the probability of not being picked for the second question was conditioned on the event that you were not picked for the first question. In Example~\ref{3wRep}, the professor sampled her students \term{with replacement}: she repeatedly sampled the entire class without regard to who she already picked.

\begin{exercise} \label{raffleOf30TicketsWWOReplacement}
Your department is holding a raffle. They sell 30 tickets and offer seven prizes. (a) They place the tickets in a hat and draw one for each prize. The tickets are sampled without replacement, i.e. the selected tickets are not placed back in the hat. What is the probability of winning a prize if you buy one ticket? (b)~What if the tickets are sampled with replacement?\footnote{(a) First determine the probability of not winning. The tickets are sampled without replacement, which means the probability you do not win on the first draw is $29/30$, $28/29$ for the second, ..., and $23/24$ for the seventh. The probability you win no prize is the product of these separate probabilities: $23/30$. That is, the probability of winning a prize is $1 - 23/30 = 7/30 = 0.233$. (b)~When the tickets are sampled with replacement, there are seven independent draws. Again we first find the probability of not winning a prize: $(29/30)^7 = 0.789$. Thus, the probability of winning (at least) one prize when drawing with replacement is 0.211.}
\end{exercise}

\begin{exercise} \label{followUpToRaffleOf30TicketsWWOReplacement}
Compare your answers in Guided Practice~\ref{raffleOf30TicketsWWOReplacement}. How much influence does the sampling method have on your chances of winning a prize?\footnote{There is about a 10\% larger chance of winning a prize when using sampling without replacement. However, at most one prize may be won under this sampling procedure.}
\end{exercise}

Had we repeated Guided Practice~\ref{raffleOf30TicketsWWOReplacement} with 300 tickets instead of 30, we would have found something interesting: the results would be nearly identical. The probability would be 0.0233 without replacement and 0.0231 with replacement. When the sample size is only a small fraction of the population (under 10\%), observations are nearly independent even when sampling without replacement.



\subsection{Independence considerations in conditional probability}

If two processes are independent, then knowing the outcome of one should provide no information about the other. We can show this is mathematically true using conditional probabilities.

\begin{exercise} \label{condProbOfRollingA1AfterOne1}
Let $X$ and $Y$ represent the outcomes of rolling two dice. (a)~What is the probability that the first die, $X$, is \resp{1}? (b)~What is the probability that both $X$ and $Y$ are \resp{1}? (c)~Use the formula for conditional probability to compute $P(Y =$ \resp{1} $| X = $ \resp{1}$)$. (d)~What is $P(Y=1)$? Is this different from the answer from part (c)? Explain.\footnote{Brief solutions: (a) $1/6$. (b) $1/36$. (c)~$\frac{P(Y = \text{ \resp{1} and}X=\text{ \resp{1}})}{P(X=\text{ \resp{1}})} = \frac{1/36}{1/6} = 1/6$. (d)~The probability is the same as in part~(c): $P(Y=1)=1/6$. The probability that $Y=1$ was unchanged by knowledge about $X$, which makes sense as $X$ and $Y$ are independent.}
\end{exercise}

We can show in Guided Practice~\ref{condProbOfRollingA1AfterOne1}(c) that the conditioning information has no influence by using the Multiplication Rule for independence processes:
\begin{eqnarray*}
P(Y=\text{\resp{1}}|X=\text{\resp{1}})
	&=& \frac{P(Y=\text{\resp{1} and}X=\text{\resp{1}})}{P(X=\text{\resp{1}})} \\
	&=& \frac{P(Y=\text{\resp{1}})\times \color{oiGB}P(X=\text{\resp{1}})}{\color{oiGB}P(X=\text{\resp{1}})} \\
	&=& P(Y=\text{\resp{1}}) \\
\end{eqnarray*}

\begin{exercise}
Ron is watching a roulette table in a casino and notices that the last five outcomes were \resp{black}. He figures that the chances of getting \resp{black} six times in a row is very small (about $1/64$) and puts his paycheck on red. What is wrong with his reasoning?\footnote{He has forgotten that the next roulette spin is independent of the previous spins. Casinos do employ this practice; they post the last several outcomes of many betting games to trick unsuspecting gamblers into believing the odds are in their favor. This is called the \term{gambler's fallacy}.}
\end{exercise}


\subsection{Checking for independent and mutually exclusive events}

If $A$ and $B$ are independent events, then the probability of $A$ being true is unchanged if $B$ is true. Mathematically, this is written as
\begin{align*}
P(A|B) &= P(A)
\end{align*}
The General Multiplication Rule states that $P(A\text{ and }B) \text{ equals}P(A | B)\times P(B)$. If $A$ and $B$ are independent events, we can replace $P(A|B)$ with $P(A)$ and the following multiplication rule applies:
\begin{align*}
P(A\text{ and }B) &= P(A)\times P(B)
\end{align*}

\begin{tipBox}{\tipBoxTitle{Checking whether two events are independent}
When checking whether two events $A$ and $B$ are independent, verify one of the following equations holds (there is no need to check both equations):
\begin{align*}
P(A|B) &= P(A)&
P(A\text{ and }B) &= P(A)\times P(B)
\end{align*}
If the equation that is checked holds true (the left and right sides are equal), $A$ and $B$ are independent. If the equation does not hold, then $A$ and $B$ are dependent.}
\end{tipBox}

\begin{example}{Are student and parent drug use independent or dependent?}\label{studentParentDrugUseIndependentExample}
We'll use the first equation above to check for independence. If student and parent drug use are independent, it must be true that
\begin{align*}
P(\text{\var{student} \resp{uses}} | \text{ \var{parent} \resp{used}}) &= P(\text{\var{student} \resp{uses}})
\end{align*}
Using Table~\ref{contTableOfParStDrugUseCopy}, we check whether equality holds in this equation.
\begin{align*}
P(\text{\var{student} \resp{uses}} | \text{ \var{parent} \resp{used}}) &\overset{?}{=} P(\text{\var{student} \resp{uses}}) \\
\frac{125}{210}=0.595 &\neq \frac{219}{445}=0.492
\end{align*}
Because the sides are not equal, student and parent drug use are dependent. That is, we estimate the probability a student uses drugs to be higher if we know that the students' parents used drugs.
\end{example}

\begin{table}[ht]
\centering
\begin{tabular}{ll rr r rr}
  && \multicolumn{2}{c}{\var{parents}} & \hspace{1cm} &  \\
  \cline{3-4}
	&& \resp{used} & \resp{not} & Total  \\
  \cline{2-5}
	& \resp{uses}     & 125 & 94 & 219 \\
\raisebox{1.5ex}[0pt]{\var{student}}	& \resp{not} \hspace{0.5cm} & 85 & 141 & 226   \\
  \cline{2-5}
	& Total & 210 & 235 & 445 \\
\end{tabular}
\caption{Contingency table summarizing the \data{drug\_\hspace{0.3mm}use} data set.}
\label{contTableOfParStDrugUseCopy}
\end{table}

\begin{exercise}Use the second equation describe above to show that student and parent drug usage are dependent.\footnote{We check for equality in the following equation:
\begin{align*}
P(\text{\var{student} \resp{uses} and \var{parent} \resp{used}})&\overset{?}{=} P(\text{\var{student} \resp{uses}})\times P(\text{\var{parent} \resp{used}})\\
\frac{125}{445}=0.281&\neq \frac{219}{445}\times \frac{210}{445}=0.232
\end{align*}
These terms are not equal, which confirms what we learned in Example~\ref{studentParentDrugUseIndependentExample}: student and parent drug use are dependent.}
\end{exercise}

If $A$ and $B$ are mutually exclusive events, then $A$ and $B$ cannot occur at the same time.  Mathematically, this is written as
\begin{align*}
P(A \text{ and } B) &= 0
\end{align*}
The General Addition Rule states that $P(A\text{ or }B) \text{ equals }P(A) + P(B) - P(A\text{ and }B)$.  If $A$ and $B$ are mutually exclusive events, we can replace $P(A \text{ and }B)$ with 0 and the following addition rule applies:
\begin{align*}
P(A\text{ or }B) &= P(A) + P(B)
\end{align*}

\begin{tipBox}{\tipBoxTitle{Checking whether two events are mutually exclusive (disjoint)}
If $A$ and $B$ are mutually exclusive events, then they cannot occur at the same time. If asked to determine if events $A$ and $B$ are mutually exclusive, verify one of the following equations holds (there is no need to check both equations):
\begin{align*}
P(A \text{ and }B) &= 0&
P(A\text{ or }B) &= P(A) + P(B)
\end{align*}
If the equation that is checked holds true (the left and right sides are equal), $A$ and $B$ are mutually exclusive. If the equation does not hold, then $A$ and $B$ are not mutually exclusive.}
\end{tipBox}

%\begin{tipBox}{\tipBoxTitle{Checking whether two events are mutually exclusive}
%If $A$ and $B$ are mutually exclusive (disjoint) events, then the probability of $A$ and $B$ occurring at the same time is zero. Mathematically, this is written as
%\begin{align*}
%P(A \text{ and}B) &= 0
%\end{align*}
%The General Addition Rule states $P(A\text{ or}B) \text{ equals}P(A) + P(B) - P(A \text{ and}B)$. If $A$ and $B$ are mutually exclusive events, we can replace $P(A \text{ and}B)$ with 0 and the following addition rule applies:
%\begin{align*}
%P(A\text{ or}B) &= P(A) + P(B)
%\end{align*}
%If asked to determine if events $A$ and $B$ are mutually exclusive, check one of the two equations above. If the equation holds, the events $A$ and $B$ are mutually exclusive. If the equation does not hold (the left and right sides are not equal), then $A$ and $B$ are not mutually exclusive.}
%\end{tipBox}

\begin{example}{Are student and parent drug use mutually exclusive?}
Looking in the table, we see that there are 125 instances where both the parents used drugs and the student uses drugs, indicating the probability of both events occurring is greater than 0. Since we have found an example where both of these events happen together, these two events are not mutually exclusive. We could more formally show this by computing the probability both events occur at the same time:
\begin{align*}
P(\text{\var{student} \resp{uses} and \var{parent} \resp{used}}) = \frac{125}{445}\neq 0
\end{align*}
Since this probability is not zero, student and parent drug use are not mutually exclusive.
\end{example}

\begin{tipBox}{\tipBoxTitle{Mutually exclusive and independent are different}
If two events are mutually exclusive, then if one is true, the other cannot be true. This implies the two events are in some way connected, meaning they must be dependent.\\
If two events are independent, then if one occurs, it is still possible for the other to occur, meaning the events are not mutually exclusive.}
\end{tipBox}

\begin{caution}{Dependent events need not be mutually exclusive.}{If two events are dependent, we can\underline{not} simply conclude they are mutually exclusive. For example, the drug use of parents and students are dependent, but they are not mutually exclusive.}
\end{caution}

\subsection{Tree diagrams}

\index{data!smallpox|)}
\index{tree diagram|(}

\termsub{Tree diagrams}{tree diagram} are a tool to organize outcomes and probabilities around the structure of the data. They are most useful when two or more processes occur in a sequence and each process is conditioned on its predecessors.

The \data{smallpox} data fit this description. We see the population as split by \var{inoculation}: \resp{yes} and \resp{no}. Following this split, survival rates were observed for each group. This structure is reflected in the tree diagram shown in Figure~\ref{smallpoxTreeDiagram}. The first branch for \var{inoculation} is said to be the \term{primary} branch while the other branches are \term{secondary}.

\begin{figure}[ht]
\centering
\includegraphics[width=0.93\textwidth]{ch_probability/figures/smallpoxTreeDiagram/smallpoxTreeDiagram}
\caption{A tree diagram of the \data{smallpox} data set.}
\label{smallpoxTreeDiagram}
\end{figure}

Tree diagrams are annotated with marginal and conditional probabilities, as shown in Figure~\ref{smallpoxTreeDiagram}. This tree diagram splits the smallpox data by \var{inoculation} into the \resp{yes} and \resp{no} groups with respective marginal probabilities 0.0392 and 0.9608. The secondary branches are conditioned on the first, so we assign conditional probabilities to these branches. For example, the top branch in Figure~\ref{smallpoxTreeDiagram} is the probability that \resp{lived} conditioned on the information that \resp{inoculated}. We may (and usually do) construct joint probabilities at the end of each branch in our tree by multiplying the numbers we come across as we move from left to right. These joint probabilities are computed using the General Multiplication Rule:
\begin{align*}
P(\text{\var{inoculated} and \resp{lived}}) &= P(\text{\var{inoculated}})\times P(\text{\resp{lived}}|\text{ \var{inoculated}}) \\
	&= 0.0392\times 0.9754 \\
	&=0.0382
\end{align*}
\begin{example}{What is the probability that a randomly selected person who was inoculated died?}This is equivalent to $P(\text{\resp{died}} | \text{\var{ inoculated}})$. This conditional probability can be found in the second branch as 0.0246.
\end{example}

\begin{example}{What is the probability that a randomly selected person lived?}There are two ways that a person could have lived:  be inoculated \emph{and} live OR not be inoculated \emph{and} live. To find this probability, we sum the two disjoint probabilities:
\begin{align*}
P(\text{\resp{lived}}) = 0.0392 \times 0.9745 + 0.9608 \times 0.8589 = 0.03824 + 0.82523 = 0.86347
\end{align*}
\end{example}

\begin{exercise}
After an introductory statistics course, 78\% of students can successfully construct tree diagrams. Of those who can construct tree diagrams, 97\% passed, while only 57\% of those students who could not construct tree diagrams passed. (a)~Organize this information into a tree diagram. {(b)~What is the probability that a student who was able to construct tree diagrams did not pass? (c)~What is the probability that a randomly selected student was able to successfully construct tree diagrams and passed? (d)~What is the probability that a randomly selected student passed?}\
\footnote{\begin{minipage}[t]{0.47\linewidth}
(a) The tree diagram is shown to the right.
(b)~$P($not pass $|$ able to construct tree diagram$) = 0.03$.
(c)~$P($able to construct tree diagrams and passed$) = P($able to construct tree diagrams$) \times P($passed $|$ able to construct tree diagrams$) = 0.78 \times 0.97 = 0.7566$.
(d)~ $P($passed$) = 0.7566 + 0.1254 = 0.8820$. \vspace{15mm} \\\
\end{minipage}
\begin{minipage}[c]{0.5\linewidth}
\includegraphics[width=\textwidth]{ch_probability/figures/treeDiagramAndPass/treeDiagramAndPass} \vspace{-25mm}
\end{minipage}}
\end{exercise}


\subsection{Bayes' Theorem}
\label{bayesTheoremSubsection}

\index{Bayes' Theorem|(}

In many instances, we are given a conditional probability of the form
\begin{align*}
P(\text{statement about variable 1} | \text{ statement about variable 2})
\end{align*}
but we would really like to know the inverted conditional probability:
\begin{align*}
P(\text{statement about variable 2} | \text{ statement about variable 1})
\end{align*}
For example, instead of wanting to know $P($lived $|$ inoculated$)$, we might want to know $P($inoculated $|$ lived$)$. This is more challenging because it cannot be read directly from the tree diagram. In these instances we use \term{Bayes' Theorem}. Let's begin by looking at a new example.

\begin{example}{In Canada, about 0.35\% of women over 40 will be diagnosed with breast cancer in any given year. A common screening test for cancer is the mammogram, but this test is not perfect. In about 11\% of patients with breast cancer, the test gives a \term{false negative}: it indicates a woman does not have breast cancer when she does have breast cancer. Similarly, the test gives a \term{false positive} in 7\% of patients who do not have breast cancer: it indicates these patients have breast cancer when they actually do not.\footnote{The probabilities reported here were obtained using studies reported at \href{http://www.breastcancer.org/symptoms/testing/new_research/20090831b.jsp}{www.breastcancer.org} and \href{http://www.ncbi.nlm.nih.gov/pmc/articles/PMC1173421/}{www.ncbi.nlm.nih.gov/pmc/articles/PMC1173421}.} If we tested a random woman over 40 for breast cancer using a mammogram and the test came back positive -- that is, the test suggested the patient has cancer -- what is the probability that the patient actually has breast cancer?}\label{probabilityOfBreastCancerGivenPositiveTestExample}
We are given sufficient information to quickly compute the probability of testing positive if a woman has breast cancer ($1.00 - 0.11 = 0.89$). However, we seek the inverted probability of cancer given a positive test result:
\begin{align*}
P(\text{has BC} | \text{ mammogram$^+$})
\end{align*}
Here, ``has BC'' is an abbreviation for the patient actually having breast cancer, and ``mammogram$^+$'' means the mammogram screening was positive, which in this case means the test suggests the patient has breast cancer. (Watch out for the non-intuitive medical language: a~\emph{positive} test result suggests the possible presence of cancer in a mammogram screening.) We can use the conditional probability formula from the previous section: $P(A|B) = \frac{P(A \text{ and } B)}{P(B)}$. Our conditional probability can be found as follows:
\begin{align*}
P(\text{has BC $|$ mammogram$^+$}) &=  \frac{P(\text{has BC and mammogram$^+$})}{P(\text{mammogram$^+$})}
\end{align*}
The probability that a mammogram is positive is as follows.
\begin{align*}
P(\text{mammogram$^+$})=P(\text{has BC and mammogram$^+$}) +  P(\text{no BC and mammogram$^+$})
\end{align*}
A tree diagram is useful for identifying each probability and is shown in Figure~\ref{BreastCancerTreeDiagram}. Using the tree diagram, we find that
\begin{align*}
P(&\text{has BC $|$ mammogram$^+$}) \\
&= \frac{P(\text{has BC and mammogram$^+$})}{P(\text{has BC and mammogram$^+$}) +  P(\text{no BC and mammogram$^+$})} \\
&= \frac{0.0035(0.89)}{0.0035(0.89)+0.9965(0.07)}\\
&= \frac{0.00312}{0.07288}\approx 0.0428
\end{align*}
That is, even if a patient has a positive mammogram screening, there is still only a~4\%~chance that she has breast cancer.
\end{example}

\begin{figure}[ht]
\centering
\includegraphics[width=0.9\textwidth]{ch_probability/figures/BreastCancerTreeDiagram/BreastCancerTreeDiagram}
\caption{Tree diagram for Example~\ref{probabilityOfBreastCancerGivenPositiveTestExample}, computing the probability a random patient who tests positive on a mammogram actually has breast cancer.}
\label{BreastCancerTreeDiagram}
\end{figure}

Example~\ref{probabilityOfBreastCancerGivenPositiveTestExample} highlights why doctors often run more tests regardless of a first positive test result. When a medical condition is rare, a single positive test isn't generally definitive.

Consider again the last equation of Example~\ref{probabilityOfBreastCancerGivenPositiveTestExample}.
Using the tree diagram, we can see that the numerator (the top of the fraction) is equal to the following product:
\begin{align*}
P(\text{has BC and mammogram$^+$}) = P(\text{mammogram$^+$} | \text{ has BC})P(\text{has BC})
\end{align*}
The denominator -- the probability the screening was positive -- is equal to the sum of probabilities for each positive screening scenario:
\begin{align*}
P(\text{\underline{\color{black}mammogram$^+$}})
	&= P(\text{\underline{\color{black}mammogram$^+$} and no BC})
		+ P(\text{\underline{\color{black}mammogram$^+$} and has BC})
\end{align*}
In the example, each of the probabilities on the right side was broken down into a product of a conditional probability and marginal probability using the tree diagram.
\begin{align*}
P(\text{mammogram$^+$})
	&= P(\text{mammogram$^+$ and no BC}) + P(\text{mammogram$^+$ and has BC}) \\
	&= P(\text{mammogram$^+$} | \text{ no BC})P(\text{no BC}) \\
			   &\qquad\qquad + P(\text{mammogram$^+$} | \text{ has BC})P(\text{has BC})
\end{align*}
We can see an application of Bayes' Theorem by substituting the resulting probability expressions into the numerator and denominator of the original conditional probability.
\begin{align*}
& P(\text{has BC} | \text{ mammogram$^+$})  \\
& \qquad= \frac{P(\text{mammogram$^+$} | \text{ has BC})P(\text{has BC})}
	{P(\text{mammogram$^+$} | \text{ no BC})P(\text{no BC}) + P(\text{mammogram$^+$} | \text{ has BC})P(\text{has BC})}
\end{align*}

\begin{termBox}{\tBoxTitle{Bayes' Theorem: inverting probabilities}
Consider the following conditional probability for variable 1 and variable 2:\vspace{-1.5mm}
\begin{align*}
P(\text{outcome $A_1$ of variable 1} | \text{ outcome $B$ of variable 2})
\end{align*}
Bayes' Theorem states that this conditional probability can be identified as the following fraction:\vspace{-1.5mm}
\begin{align}
\frac{P(B | A_1) P(A_1)}
	{P(B | A_1) P(A_1) + P(B | A_2) P(A_2) + \cdots + P(B | A_k) P(A_k)}
	\label{equationOfBayesTheorem}
\end{align}
where $A_2$, $A_3$, ..., and $A_k$ represent all other possible outcomes of the first variable.}\index{Bayes' Theorem|textbf}
\end{termBox}

Bayes' Theorem is just a generalization of what we have done using tree diagrams. The formula need not be memorized, since it can always be derived using a tree diagram:
\begin{itemize}
\setlength{\itemsep}{0mm}
\item The numerator identifies the probability of getting both $A_1$ and $B$.
\item The denominator is the overall probability of getting $B$.  Traverse each branch of the tree diagram that ends with event $B$.  Add up the required products.
\end{itemize}

\begin{exercise} \label{exerciseForParkingLotOnCampusBeingFullAndWhetherOrNotThereIsASportingEvent}
Jose visits campus every Thursday evening. However, some days the parking garage is full, often due to college events. There are academic events on 35\% of evenings, sporting events on 20\% of evenings, and no events on 45\% of evenings. When there is an academic event, the garage fills up about 25\% of the time, and it fills up 70\% of evenings with sporting events. On evenings when there are no events, it only fills up about 5\% of the time. If Jose comes to campus and finds the garage full, what is the probability that there is a sporting event? Use a tree diagram to solve this problem.\footnote{\begin{minipage}[t]{0.47\textwidth}
The tree diagram, with three primary branches, is shown to the right. We want
\begin{align*}
P(&\text{sporting event} | \text{garage full}) \\
&= \frac{P(\text{sporting event and garage full})}{P(\text{garage full})} \\
&=\frac{0.14}{0.0875 + 0.14 + 0.0225} = 0.56.
\end{align*}
If the garage is full, there is a 56\% probability that there is a sporting event. \vspace{0.1mm} \\\
\end{minipage}
\begin{minipage}[c]{0.5\textwidth}
\includegraphics[width=\textwidth]{ch_probability/figures/treeDiagramGarage/treeDiagramGarage}\vspace{-30mm}
\end{minipage}}
\end{exercise}

The last several exercises offered a way to update our belief about whether there is a sporting event, academic event, or no event going on at the school based on the information that the parking lot was full. This strategy of \emph{updating beliefs} using Bayes' Theorem is actually the foundation of an entire section of statistics called \term{Bayesian statistics}. While Bayesian statistics is very important and useful, we will not have time to cover it in this book.

\index{Bayes' Theorem|)}
\index{tree diagram|)}
\index{conditional probability|)}
\index{probability|)}




\section{The binomial formula}

\begin{example}{Suppose we randomly selected four individuals to participate in the ``shock" study. What is the chance exactly one of them will be a success?  Let's call the four people Allen ($A$), Brittany ($B$), Caroline ($C$), and Damian ($D$) for convenience. Also, suppose 35\% of people are successes as in the previous version of this example.}\label{oneRefuser}
Let's consider a scenario where one person refuses:
\begin{eqnarray*}
&&P(A=\text{\resp{refuse}},\text{ }B=\text{\resp{shock}},\text{ }C=\text{\resp{shock}},\text{ }D=\text{\resp{shock}}) \\
 &&\quad =  P(A=\text{\resp{refuse}})\ P(B=\text{\resp{shock}})\ P(C=\text{\resp{shock}})\ P(D=\text{\resp{shock}}) \\
 &&\quad =  (0.35)  (0.65)  (0.65)  (0.65) = (0.35)^1 (0.65)^3 = 0.096
\end{eqnarray*}
But there are three other scenarios: Brittany, Caroline, or Damian could have been the one to refuse. In each of these cases, the probability is again $(0.35)^1(0.65)^3$. These four scenarios exhaust all the possible ways that exactly one of these four people could refuse to administer the most severe shock, so the total probability is $4\times(0.35)^1(0.65)^3 = 0.38$.
\end{example}

\begin{exercise}
Verify that the scenario where Brittany is the only one to refuse to give the most severe shock has probability $(0.35)^1(0.65)^3$. \footnote{$P(A=\text{\resp{shock}},\text{ }B=\text{\resp{refuse}},\text{ }C=\text{\resp{shock}},\text{ }D=\text{\resp{shock}}) = (0.65)(0.35)(0.65)(0.65) = (0.35)^1(0.65)^3$.}
\end{exercise}


\subsection{Understanding the formula}
To solve the scenario outlined in Example~\ref{oneRefuser} we use what is called the \term{Binomal Formula}.  The binomal formula gives the probability of having $k$ successes in $n$ independent trials where probability of an individual success in one trial is $p$ (in Example~\ref{oneRefuser}, $n=4$, $k=1$, $p=0.35$). In order to develop this formula, we reexamine each part of the example.

There were four individuals who could have been the one to refuse, and each of these four scenarios had the same probability. Thus, we could identify the final probability as
\begin{eqnarray}
[\text{\# of scenarios}] \times P(\text{single scenario})
\label{genBinomialFormula}
\end{eqnarray}
The first component of this equation is the number of ways to arrange the $k=1$ successes among the $n=4$ trials. The second component is the probability of any of the four (equally probable) scenarios.

Consider $P($single scenario$)$ under the general case of $k$ successes and $n-k$ failures in the $n$ trials. In any such scenario, we apply the Multiplication Rule for independent events:
\begin{eqnarray*}
p^k(1-p)^{n-k}
\end{eqnarray*}
This is our general formula for $P($single scenario$)$.

Secondly, we introduce a general formula for the number of ways to choose $k$ successes in $n$ trials, i.e. arrange $k$ successes and $n-k$ failures:
\begin{eqnarray*}
{n\choose k} = \frac{n!}{k!(n-k)!}
\end{eqnarray*}
The quantity ${n\choose k}$ is read \term{n choose k}.\footnote{Other notation for $n$ choose $k$ includes $_nC_k$, $C_n^k$, and $C(n,k)$.} The exclamation point notation (e.g. $k!$) denotes a \term{factorial}\label{factorialDefinitionInTheBinomialSection} expression.
\begin{eqnarray*}
&& 0! = 1 \label{zeroFactorial} \\
&& 1! = 1 \\
&& 2! = 2\times1 = 2 \\
&& 3! = 3\times2\times1 = 6 \\
&& 4! = 4\times3\times2\times1 = 24 \\
&& \vdots \\
&& n! = n\times(n-1)\times...\times3\times2\times1
\end{eqnarray*}
Using the formula, we can compute the number of ways to choose $k=1$ successes in $n=4$ trials:
\begin{eqnarray*}
{4 \choose 1} = \frac{4!}{1!(4-1)!} =  \frac{4!}{1!3!} 
	= \frac{4\times3\times2\times1}{(1)(3\times2\times1)} = 4
\end{eqnarray*}
This result is exactly what we found by carefully thinking of each possible scenario in Example~\ref{oneRefuser}.

Substituting $n$ choose $k$ for the number of scenarios and $p^k(1-p)^{n-k}$ for the single scenario probability in Equation~(\ref{genBinomialFormula}) yields the general binomial formula.

%\Comment{changed ``binomial distribution" to ``binomial formula" and deleted mean, sd, and variance from termbox}

\begin{termBox}{\tBoxTitle{Binomial formula} Suppose the probability of a single trial being a success is $p$. Then the probability of observing exactly $k$ successes in $n$ independent trials is given by\vspace{-1mm}
\begin{eqnarray}
{n\choose k}p^k(1-p)^{n-k} = \frac{n!}{k!(n-k)!}p^k(1-p)^{n-k}
\label{binomialFormula}
\end{eqnarray}
}
\end{termBox}

\subsection{When and how to apply the formula}

\begin{tipBox}{\tipBoxTitle{Is it binomial? Four conditions to check.\label{isItBinomialTipBox}}
(1) The trials are independent. \\
(2) The number of trials, $n$, is fixed. \\
(3) Each trial outcome can be classified as a \emph{success} or \emph{failure}. \\
(4) The probability of a success, $p$, is the same for each trial.}
\end{tipBox}

\begin{example}{What is the probability that 3 of 8 randomly selected students will refuse to administer the worst shock, i.e. 5 of 8 will?}
We would like to apply the binomial model, so we check our conditions. The number of trials is fixed ($n=8$) (condition 2) and each trial outcome can be classified as a success or failure (condition 3). Because the sample is random, the trials are independent (condition~1) and the probability of a success is the same for each trial (condition~4).

In the outcome of interest, there are $k=3$ successes in $n=8$ trials, and the probability of a success is $p=0.35$. So the probability that 3 of 8 will refuse is given by
\begin{eqnarray*}
{ 8 \choose 3}(0.35)^3(1-0.35)^{8-3}
	&=& \frac{8!}{3!(8-3)!}(0.35)^3(1-0.35)^{8-3} \\
	&=& \frac{8!}{3!5!}(0.35)^3(0.65)^5
\end{eqnarray*}
Dealing with the factorial part:
\begin{eqnarray*}
\frac{8!}{3!5!} = \frac{8\times7\times6\times5\times4\times3\times2\times1}{(3\times2\times1)(5\times4\times3\times2\times1)} = \frac{8\times7\times6}{3\times2\times1} = 56
\end{eqnarray*}
Using $(0.35)^3(0.65)^5 \approx 0.005$, the final probability is about $56*0.005 = 0.28$.
\end{example}

\begin{tipBox}{\tipBoxTitle{computing binomial probabilities}
The first step in using the binomial model is to check that the model is appropriate. If it is, the next step is to identify $n$, $p$, and $k$. The final step is to apply the formulas and interpret the results.}
\end{tipBox}

\begin{exercise}
The probability that a random smoker will develop a severe lung condition in his or her lifetime is about $0.3$. If you have 4 friends who smoke, are the conditions for the binomial model satisfied?\footnote{One possible answer: if the friends know each other, then the independence assumption is probably not satisfied. For example, acquaintances may have similar smoking habits.}
\end{exercise}

\begin{exercise}
\label{noMoreThanOneFriendWSevereLungCondition}%
Suppose these four friends do not know each other and we can treat them as if they were a random sample from the population. Is the binomial model appropriate? What is the probability that (a) none of them will develop a severe lung condition? (b) One will develop a severe lung condition? (c) That no more than one will develop a severe lung condition?\footnote{To check if the binomial model is appropriate, we must verify the conditions. (i) Since we are supposing we can treat the friends as a random sample, they are independent. (ii) We have a fixed number of trials ($n=4$). (iii) Each outcome is a success or failure. (iv) The probability of a success is the same for each trials since the individuals are like a random sample ($p=0.3$ if we say a ``success'' is someone getting a lung condition, a morbid choice). Compute parts (a) and (b) from the binomial formula in Equation~\eqref{binomialFormula}: $P(0) =  {4 \choose 0} (0.3)^0 (0.7)^4 = 1\times1\times0.7^4 = 0.2401$, $P(1) = {4 \choose 1} (0.3)^1(0.7)^{3} = 0.4116$. Note: $0!=1$, as shown on page~\pageref{zeroFactorial}. Part (c) can be computed as the sum of parts (a) and (b): $P(0) + P(1) = 0.2401 + 0.4116 = 0.6517$. That is, there is about a 65\% chance that no more than one of your four smoking friends will develop a severe lung condition.}
\end{exercise}

\begin{exercise}
What is the probability that at least 2 of your 4 smoking friends will develop a severe lung condition in their lifetimes?\footnote{The complement (no more than one will develop a severe lung condition) as computed in Guided Practice~\ref{noMoreThanOneFriendWSevereLungCondition} as 0.6517, so we compute one minus this value: 0.3483.}
\end{exercise}

\begin{exercise}
Suppose you have 7 friends who are smokers and they can be treated as a random sample of smokers. What is the probability that at most 2 of your 7 friends will develop a severe lung condition.\footnote{$P($0, 1, or 2 develop severe lung condition$) = P(k=0) + P(k=1)+P(k=2) = 0.6471$.}
\end{exercise}

Below we consider the first term in the binomial probability, $n$ choose $k$ under some special scenarios.

\begin{exercise}
Why is it true that ${n \choose 0}=1$ and ${n \choose n}=1$ for any \mbox{number~$n$}?\footnote{Frame these expressions into words. How many different ways are there to arrange 0 successes and $n$ failures in $n$ trials? (1 way.) How many different ways are there to arrange $n$ successes and 0 failures in $n$ trials? (1 way.)}
\end{exercise}

\begin{exercise}
How many ways can you arrange one success and $n-1$ failures in $n$ trials? How many ways can you arrange $n-1$ successes and one failure in $n$ trials?\footnote{One success and $n-1$ failures: there are exactly $n$ unique places we can put the success, so there are $n$ ways to arrange one success and $n-1$ failures. A similar argument is used for the second question. Mathematically, we show these results by verifying the following two equations:
\begin{eqnarray*}
{n \choose 1} = n, \qquad {n \choose n-1} = n
\end{eqnarray*}}
\end{exercise}

\begin{example}{There are 13 marbles in a bag.  4 are blue and 9 are red.  Randomly draw 5 marbles \emph{without replacement}.  Find the probability you get exactly 3 blue marbles.}Because the probability of success $p$ is not the same for each trial, we cannot use the binomial formula.  However, we can use the same logic to arrive at the following answer.  
\begin{align*}
P(x = 3) &= (\text{\# of combinations with 3 blue})\times P(\text{3 blue and 2 red in a \emph{specific} order}) \\
&={5\choose 3}\times P(\text{RRRBB}) \\
&= {5\choose 3}\left(\frac{4}{13}\times \frac{3}{12}\times \frac{2}{11} \times \frac{9}{10} \times \frac{8}{9}\right) \\
&= 0.1112
\end{align*}
\end{example}


\subsection{Calculator: binomial probabilities}
\label{calculatorBinomial}

\begin{termBox}{\tBoxTitle{TI-83/84: Computing the binomial coefficient: ${n\choose k}$\vspace{1mm}}
Use \textbf{MATH, PRB, nCr} to evaluate $n$ choose $r$. Here r and k are different letters for the same quantity. e.g.: 5 nCr 3 means 5 choose 3.
\begin{enumerate}
\setlength{\itemsep}{0mm}
\item Type the value of n.
\item Select MATH.
\item Right arrow to PRB.
\item Choose 3:nCr.
\item Type the value of k.
\item Hit ENTER.
\end{enumerate}}
\end{termBox}

\Comment{D - TI-83/84 for the two boxes below, or just TI-84?}

\begin{termBox}{\tBoxTitle{TI-84: Computing binomial formula: $P(x = k)={n\choose k}p^k(1-p)^{n-k}$\vspace{1mm}} 
Use \textbf{2ND VARS, binompdf} to evaluate the probability of \emph{exactly} $k$ occurrences out of $n$ independent trials of an event with probability $p$. 
\begin{enumerate}
\setlength{\itemsep}{0mm}
\item Select 2ND VARS (i.e. DISTR)
\item Choose A:binompdf  (use the down arrow).
\item Let trials be $n$.
\item Let p be $p$
\item Let x value be $k$.
\item Select Paste and hit ENTER.\vspace{-1.5mm}
\begin{itemize}
\item[TI-83: ] Do steps 1 - 2, then enter $n$, $p$, and $k$ separated by commas as follows:  binompdf($n$, $p$, $k$). Then hit ENTER. 
\end{itemize}
\end{enumerate}
}
\end{termBox}

\begin{termBox}{\tBoxTitle{TI-84: Computing $P(x \le k)= {n\choose 0}p^0(1-p)^{n-0} + ... + {n\choose k}p^k(1-p)^{n-k}$} 
Use \textbf{2ND VARS, binomcdf} to evaluate the cumulative probability of \emph{at most} $k$ occurrences out of $n$ independent trials of an event with probability $p$. 
\begin{enumerate}
\setlength{\itemsep}{0mm}
\item Select 2ND VARS (i.e. DISTR)
\item Choose B:binomcdf  (use the down arrow).
\item Let trials be $n$.
\item Let p be $p$
\item Let x value be $k$.
\item Select Paste and hit ENTER.\vspace{-1.5mm}
\begin{itemize}
\item[TI-83: ] Do steps 1 - 2, then enter $n$, $p$, and $k$ separated by commas as follows: binomcdf($n$, $p$, $k$). Then hit ENTER.
\end{itemize}
\end{enumerate}
}
\end{termBox}

\begin{exercise}Find the number of ways of arranging 3 blue marbles and 2 red marbles.\footnote{Use 5 nCr 3 to get 10. }
\end{exercise}

\begin{exercise}There are 13 marbles in a bag.  4 are blue and 9 are red.  Randomly draw 5 marbles \emph{with replacement}.  Find the probability you get exactly 3 blue marbles.\footnote{Use binompdf(5, 4/13, 3) to get 0.1396.}
\end{exercise}

\begin{exercise}There are 13 marbles in a bag.  4 are blue and 9 are red.  Randomly draw 5 marbles \emph{with replacement}.  Find the probability you get \emph{at most} 3 blue marbles (i.e. less than or equal to 3 blue marbles).\footnote{Use binomcdf(5, 4/13, 3) to get 0.9662.}
\end{exercise}




%_________________
\section{Random variables}
\label{randomVariablesSection}

\index{random variable|(}

\begin{example}{Two books are assigned for a statistics class: a textbook and its corresponding study guide. The university bookstore determined 20\% of enrolled students do not buy either book, 55\% buy the textbook only, and 25\% buy both books, and these percentages are relatively constant from one term to another. If~there are 100 students enrolled, how many books should the bookstore expect to sell to this class?}\label{bookStoreSales}
Around 20 students will not buy either book (0 books total), about 55 will buy one book (55 books total), and approximately 25 will buy two books (totaling 50 books for these 25 students). The bookstore should expect to sell about 105 books for this class.
\end{example}

\begin{exercise}
Would you be surprised if the bookstore sold slightly more or less than 105 books?\footnote{If they sell a little more or a little less, this should not be a surprise. Hopefully Chapter~\ref{summarizingData} helped make clear that there is natural variability in observed data. For example, if we would flip a coin 100 times, it will not usually come up heads exactly half the time, but it will probably be close.}
\end{exercise}

\begin{example}{The textbook costs \$137 and the study guide \$33. How much revenue should the bookstore expect from this class of 100 students?}\label{bookStoreRev}
About 55 students will just buy a textbook, providing revenue of
\begin{eqnarray*}
\$137 \times  55 = \$7,535
\end{eqnarray*}
The roughly 25 students who buy both the textbook and the study guide would pay a total of
\begin{eqnarray*}
(\$137 + \$33) \times  25 = \$170 \times  25 = \$4,250
\end{eqnarray*}
Thus, the bookstore should expect to generate about $\$7,535 + \$4,250 = \$11,785$ from these 100 students for this one class. However, there might be some \emph{sampling variability} so the actual amount may differ by a little bit.
\end{example}

\begin{figure}[hhh]
\centering
\includegraphics[width=0.65\textwidth]{ch_probability/figures/bookCostDist/bookCostDist}
\caption{Probability distribution for the bookstore's revenue from a single student. The distribution balances on a triangle representing the average revenue per student.}
\label{bookCostDist}
\end{figure}

\begin{example}{What is the average revenue per student for this course?}\label{revFromStudent}
The expected total revenue is \$11,785, and there are 100 students. Therefore the expected revenue per student is $\$11,785/100 =  \$117.85$.
\end{example}


\subsection{Probability distributions}


A \term{probability distribution} is a table of all disjoint outcomes and their associated probabilities. Table~\ref{diceProb} shows the probability distribution for the sum of two dice.

\begin{termBox}{\tBoxTitle{Rules for probability distributions}
A probability distribution is a list of the possible outcomes with corresponding probabilities that satisfies three rules: \vspace{-2mm}
\begin{enumerate}
\setlength{\itemsep}{0mm}
\item The outcomes listed must be disjoint.
\item Each probability must be between 0 and 1.
\item The probabilities must total 1. \vspace{1mm}
\end{enumerate}}
\end{termBox}

\begin{exercise}\label{usHouseholdIncomeDistsExercise}
Table~\ref{usHouseholdIncomeDists} suggests three distributions for household income in the United States. Only one is correct. Which one must it be? What is wrong with the other two?\footnote{The probabilities of (a) do not sum to 1. The second probability in (b) is negative. This leaves (c), which sure enough satisfies the requirements of a distribution. One of the three was said to be the actual distribution of US household incomes, so it must be (c).}
\end{exercise}

\begin{table}
\centering
\begin{tabular}{l ccc ccc ccc cc}
  \hline
  \ \vspace{-3mm} \\
Dice sum\vspace{0.3mm} & 2 & 3 & 4 & 5 & 6 & 7 & 8 & 9 & 10 & 11 & 12  \\
Probability & $\frac{1}{36}$ & $\frac{2}{36}$ & $\frac{3}{36}$ & $\frac{4}{36}$ & $\frac{5}{36}$ & $\frac{6}{36}$ & $\frac{5}{36}$ & $\frac{4}{36}$ & $\frac{3}{36}$ & $\frac{2}{36}$ & $\frac{1}{36}$\vspace{1mm} \\
   \hline
\end{tabular}
\caption{Probability distribution for the sum of two dice.}
\label{diceProb}
\ \\[5mm]
\begin{tabular}{r | rr rr}
  \hline
Income range (\$1000s) & 0-25    & 25-50    & 50-100     & 100+    \\
  \hline
(a)\hspace{0.2mm}	 & 0.18 & 0.39 & 0.33 & 0.16 \\
(b)				 & 0.38 & -0.27 & 0.52 & 0.37 \\
(c)\hspace{0.2mm}	 & 0.28 & 0.27 & 0.29 & 0.16 \\
  \hline
\end{tabular}
\caption{Proposed distributions of US household incomes (Guided Practice~\ref{usHouseholdIncomeDistsExercise}).}
\label{usHouseholdIncomeDists}
\end{table}

Chapter~\ref{summarizingData} emphasized the importance of plotting data to provide quick summaries. Probability distributions can also be summarized in a histogram or bar plot. The probability distribution for the sum of two dice is shown in Table~\ref{diceProb} and its histogram is plotted in Figure~\ref{diceSumDist}. The distribution of US household incomes is shown in Figure~\ref{usHouseholdIncomeDistBar} as a bar plot. The presence of the 100+ category makes it difficult to represent it with a regular histogram.\footnote{It is also possible to construct a distribution plot when income is not artificially binned into four groups. Density histograms for \emph{continuous} distributions are considered in Section~\ref{contDist}.}

\begin{figure}
\centering
\includegraphics[width=0.73\textwidth]{ch_probability/figures/diceSumDist/diceSumDist}
\caption{A histogram for the probability distribution of the sum of two~dice.}
\label{diceSumDist}
\end{figure}

\begin{figure}
\centering
\includegraphics[width=0.68\textwidth]{ch_probability/figures/usHouseholdIncomeDistBar/usHouseholdIncomeDistBar}
\caption{A bar graph for the probability distribution of US household income. Because it is artificially separated into four unequal bins, this graph fails to show the shape or skew of the distribution.}
\label{usHouseholdIncomeDistBar}
\end{figure}

In these bar plots, the bar heights represent the probabilities of outcomes. If the outcomes are numerical and discrete, it is usually (visually) convenient to make a histogram, as in the case of the sum of two dice. Another example of plotting the bars at their respective locations is shown in Figure~\ref{bookCostDist}.


\subsection{Expectation}

\index{expectation|(}

We call a variable or process with a numerical outcome a \term{random variable}, and we usually represent this random variable with a capital letter such as $X$, $Y$, or $Z$. The amount of money a single student will spend on her statistics books is a random variable, and we represent it by $X$.

\begin{termBox}{\tBoxTitle{Random variable}
A random process or variable with a numerical outcome.}
\end{termBox}

The possible outcomes of $X$ are labeled with a corresponding lower case letter $x$ and subscripts. For example, we write $x_1=\$0$, $x_2=\$137$, and $x_3=\$170$, which occur with probabilities $0.20$, $0.55$, and $0.25$. The distribution of $X$ is summarized in Figure~\ref{bookCostDist} and Table~\ref{statSpendDist}.

\begin{table}
\centering
\begin{tabular}{l ccc r}
\hline
$i$	  & 1 & 2 & 3  & Total\\
\hline
$x_i$ & \$0 & \$137 & \$170 & --\\
$p_i$ & 0.20 & 0.55 & 0.25 & 1.00 \\
\hline
\end{tabular}
\caption{The probability distribution for the random variable $X$, representing the bookstore's revenue from a single student. We use $p_i$ to represent the probability of $x_i$.}
\label{statSpendDist}
\end{table}

We computed the average outcome of $X$ as \$117.85 in Example~\ref{revFromStudent}. We call this average the \term{expected value} of $X$, denoted by $E(X)$\index{EX@$E(X)$}\marginpar[\raggedright\vspace{-3mm}

$E(X)$\vspace{1mm}\\\footnotesize Expected\\value of $X$]{\raggedright\vspace{-3mm}

$E(X)$\vspace{1mm}\\\footnotesize Expected\\value of $X$}. The expected value of a random variable is computed by adding each outcome weighted by its probability:
\begin{align*}
E(X) &= 0 \times  P(X=0) + 137 \times  P(X=137) + 170 \times  P(X=170) \\
	&= 0 \times  0.20 + 137 \times  0.55 + 170 \times  0.25 = 117.85
\end{align*}

\begin{termBox}{\tBoxTitle{Expected value of a Discrete Random Variable}
If $X$ takes outcomes $x_1$, $x_2$, ..., $x_n$ with probabilities $p_1$, $p_2$, ..., $p_n$, the expected value of $X$ is the sum of each outcome multiplied by its corresponding probability:
\begin{align}
E(X) = \mu_x &= x_1\times p_1 + x_2\times p_2 + \cdots + x_n\times p_n \notag \\
	&= \sum_{i=1}^{n}(x_i\times p_i)
\end{align}
}
\end{termBox}

The expected value for a random variable represents the average outcome. For example, $E(X)=117.85$ represents the average amount the bookstore expects to make from a single student, which we could also write as $\mu=117.85$. While the bookstore will make more than this on some students and less than this on other students, the average of many randomly selected students will be near \$117.85.

It is also possible to compute the expected value of a continuous random variable (see Section~\ref{contDist}). However, it requires a little calculus and we save it for a later class.\footnote{$\mu_x = \int xf(x)dx$ where $f(x)$ represents a function for the density curve.}

In physics, the expectation holds the same meaning as the center of gravity. The distribution can be represented by a series of weights at each outcome, and the mean represents the balancing point. This is represented in Figures~\ref{bookCostDist} and~\ref{bookWts}. The idea of a center of gravity also expands to continuous probability distributions. Figure~\ref{contBalance} shows a continuous probability distribution balanced atop a wedge placed at the mean.

\begin{figure}
\centering
\includegraphics[width=0.72\textwidth]{ch_probability/figures/bookWts/bookWts}
\caption{A weight system representing the probability distribution for $X$. The string holds the distribution at the mean to keep the system balanced.}
\label{bookWts}
\end{figure}

\begin{figure}
\centering
\includegraphics[width=0.68\textwidth]{ch_probability/figures/contBalance/contBalance}
\caption{A continuous distribution can also be balanced at its mean.}
\label{contBalance}
\end{figure}

\index{expectation|)}


\subsection{Variability in random variables}

Suppose you ran the university bookstore. Besides how much revenue you expect to generate, you might also want to know the volatility (variability) in your revenue.

The \indexthis{variance}{variance} and \indexthis{standard deviation}{standard deviation} can be used to describe the variability of a random variable. Section~\ref{variability}
introduced a method for finding the variance and standard deviation for a data set. We first computed deviations from the mean ($x_i - \mu$), squared those deviations, and took an average to get the variance. In the case of a random variable, we again compute squared deviations. However, we take their sum weighted by their corresponding probabilities, just like we did for the expectation. This weighted sum of squared deviations equals the variance, and we calculate the standard deviation by taking the square root of the variance, just as we did in Section~\ref{variability}.

\begin{termBox}{\tBoxTitle{Variance and Standard Deviation of a discrete random variable}
If $X$ takes outcomes $x_1$, $x_2$, ..., $x_n$ with probabilities $p_1$,  $p_2$, ..., $p_n$ and expected value $\mu_x=E(X)$, then to find the standard deviation of $X$, we first find the variance and then take its square root.
\begin{align}
Var(X) = \sigma^2_x &= (x_1-\mu_x)^2\times p_1 + (x_2-\mu_x)^2\times p_2 + \cdots +  (x_n-\mu_x)^2\times p_n \notag \\
	&= \sum_{i=1}^{n} (x_i - \mu_x)^2 \times p_i \notag \\
SD(X) = \sigma_x &= \sqrt{ \sum_{i=1}^{n} (x_i - \mu_x)^2 \times p_i}
\end{align}
}
\end{termBox}

Just as it is possible to compute the mean of a continuous random variable using calculus, we can also use calculus to compute the variance.\footnote{$\sigma^2_x = \int (x - \mu_x)^2f(x)dx$ where $f(x)$ represents a function for the density curve.} However, this topic is beyond the scope of the AP exam.

\begin{example}{Compute the expected value, variance, and standard deviation of $X$, the revenue of a single statistics student for the bookstore.}
It is useful to construct a table that holds computations for each outcome separately, then add up the results.
\begin{center}
\begin{tabular}{l rrr r}
\hline
$i$ & 1 & 2&  3& Total \\
\hline
$x_i$ & \$0 & \$137 & \$170 &  \\
$p_i$ & 0.20 & 0.55 & 0.25 &  \\
\hline
$x_i \times  p_i$ & 0 & 75.35 & 42.50 & 117.85 \\
\hline
\end{tabular}
\end{center}

Thus, the expected value is $\mu=117.85$, which we computed earlier. The variance can be constructed by extending this table:
\begin{center}
\begin{tabular}{l rrr r}
\hline
$i$ & 1 & 2 & 3 & Total \\
\hline
$x_i$ & \$0 & \$137 & \$170 &  \\
$p_i$ & 0.20 & 0.55 & 0.25 &  \\
\hline
$x_i \times  p_i$ & 0 & 75.35 & 42.50 & 117.85 \\
$x_i - \mu_{\mbox{\tiny\itshape X}}$ & -117.85 & 19.15 & 52.15 &  \\
$(x_i-\mu_{\mbox{\tiny\itshape X}})^2$ & 13888.62 &  366.72 & 2719.62 &  \\
$(x_i-\mu_{\mbox{\tiny\itshape X}})^2\times p_i$ & 2777.7 & 201.7 & 679.9 & 3659.3 \\
\hline
\end{tabular}
\end{center}
The variance of $X$ is $\sigma_x^2 = 3659.3$, which means the standard deviation is $\sigma_x = \sqrt{3659.3} = \$60.49$.
\end{example}

\begin{exercise}
The bookstore also offers a chemistry textbook for \$159 and a book supplement for \$41. From past experience, they know about 25\% of chemistry students just buy the textbook while 60\% buy both the textbook and supplement.\footnote{(a) 100\% - 25\% - 60\% = 15\% of students do not buy any books for the class. Part~(b) is represented by the first two lines in the table below. The expectation for part~(c) is given as the total on the line $y_i\times p_i$. The result of part~(d) is the square-root of the variance listed on in the total on the last line: $\sigma_{\mbox{\tiny\itshape Y}} = \sqrt{Var(Y)} = \$69.28$.
\begin{center}
\begin{tabular}{rrrrr}
$i$ (scenario) & 1 (\resp{noBook}) & 2 (\resp{textbook}) & 3 (\resp{both}) & Total \\
  \hline
$y_i$ & 0.00 & 159.00 & 200.00 &  \\
$p_i$ & 0.15 & 0.25 & 0.60 & \\
\hline
$y_i\times p_i$ & 0.00 & 39.75 & 120.00 & $E(Y) = 159.75$\\
$y_i-\mu_{\mbox{\tiny\itshape Y}}$ & -159.75 & -0.75 & 40.25 & \\
$(y_i-\mu_{\mbox{\tiny\itshape Y}})^2$ & 25520.06 & 0.56 & 1620.06 & \\
$(y_i-\mu_{\mbox{\tiny\itshape Y}})^2\times p_i$ & 3828.0 & 0.1 & 972.0 & $Var(Y) \approx 4800$ \\
\hline
\end{tabular}
\end{center}}
\begin{enumerate}
\item[(a)] What proportion of students don't buy either book? Assume no students buy the supplement without the textbook.
\item[(b)] Let $Y$ represent the revenue from a single student. Write out the probability distribution of $Y$, i.e. a table for each outcome and its associated probability.
\item[(c)] Compute the expected revenue from a single chemistry student.
\item[(d)] Find the standard deviation to describe the variability associated with the revenue from a single student.
\end{enumerate}
\end{exercise}


\subsection{Linear transformations of a random variable}

Let X be a random variable that represents how many books per student a textbook company sells. The probability distribution of X is given in the following table.

\begin{center}
\begin{tabular}{l rrr}
\hline
$x_i$ & 1 & 2 & 3  \\
\hline
$p_i$  & 0.6 & 0.3 & 0.1 \\
\hline
\end{tabular}
\end{center}

Using the methods of the previous section we can find that the mean $\mu_x = 1.5$ and the standard deviation $\sigma_x = 0.67$.
Suppose that the revenue the textbook company makes per student is \$150 and that each book has a fixed cost of \$30. The profit function, then, is $150X - 30$, where X is the number of books sold. To calculate the mean and standard deviation for the profit of the textbook company, we could define a new variable $Y$ as follows:

\begin{align*}
Y = 150X - 30
\end{align*}

\begin{exercise}Verify that the distribution of $Y$ is given by the table below.\footnote{$150 \times 1 - 30 = 120$;  $150 \times 2 - 30 = 270$; $150 \times 3 - 30 = 420$}

\begin{center}
\begin{tabular}{l rrr}
\hline
$y_i$ & \$120 & \$270 & \$420  \\
\hline
$p_i$ & 0.6 & 0.3 & 0.1 \\
\hline
\end{tabular}
\end{center}

\end{exercise}
Using this new table, we can compute the mean and standard deviation of the textbook company's profit. However, because Y is a linear transformation of X, we can use the properties from Section~\ref{linearTransformationOfData}. Recall that multiplying every X by 150 multiplies both the mean and standard deviation by 150. Subtracting 30 only subtracts 30 from the mean, not the standard deviation. Therefore,
\begin{align*}
\mu_{150X-30}&=E(150X-30) 
	& \sigma_{150X-30}&=SD(150X-30)\\
&= 150\times E(X) - 30
	& &= 150\times SD(X) \\
&= 150\times 1.5 - 30
	& &= 150\times 0.67 \\
&= 195
	& &= 100.5
\end{align*}
For a randomly selected student, the textbook company can expect to make \$195 dollars, with a standard deviation of \$100.50.

\begin{termBox}{\tBoxTitle{Linear transformations of a random variable}
If $X$ is a random variable, then a linear transformation is given by $aX + b$, where $a$ and $b$ are some fixed numbers.
\begin{align*}
E(aX+b) &= a\times E(X) + b
&
SD(aX+b) &= \lvert a\rvert \times SD(X)
\end{align*}
}
\end{termBox}

\subsection{Linear combinations of random variables}

So far, we have thought of each variable as being a complete story in and of itself. Sometimes it is more appropriate to use a combination of variables. For instance, the amount of time a person spends commuting to work each week can be broken down into several daily commutes. Similarly, the total gain or loss in a stock portfolio is the sum of the gains and losses in its components.

\begin{example}{John travels to work five days a week. We will use $X_1$ to represent his travel time on Monday, $X_2$ to represent his travel time on Tuesday, and so on. Write an equation using $X_1$, ..., $X_5$ that represents his travel time for the week, denoted by $W$.}
His total weekly travel time is the sum of the five daily values:
$$ W = X_1 + X_2 + X_3 + X_4 + X_5 $$
Breaking the weekly travel time $W$ into pieces provides a framework for understanding each source of randomness and is useful for modeling $W$.
\end{example}

\begin{example}{It takes John an average of 18 minutes each day to commute to work. What would you expect his average commute time to be for the week?}
We were told that the average (i.e. expected value) of the commute time is 18 minutes per day: $E(X_i) = 18$. To get the expected time for the sum of the five days, we can add up the expected time for each individual day:
\begin{align*}
E(W) &= E(X_1 + X_2 + X_3 + X_4 + X_5) \\
	&= E(X_1) + E(X_2) + E(X_3) + E(X_4) + E(X_5) \\
	&= 18 + 18 + 18 + 18 + 18 = 90\text{ minutes}
\end{align*}
The expectation of the total time is equal to the sum of the expected individual times. More generally, the expectation of a sum of random variables is always the sum of the expectation for each random variable.
\end{example}

\begin{exercise} \label{elenaIsSellingATVAndBuyingAToasterOvenAtAnAuction}
Elena is selling a TV at a cash auction and also intends to buy a toaster oven in the auction. If $X$ represents the profit for selling the TV and $Y$ represents the cost of the toaster oven, write an equation that represents the net change in Elena's cash.\footnote{She will make $X$ dollars on the TV but spend $Y$ dollars on the toaster oven: $X-Y$.}
\end{exercise}

\begin{exercise}
Based on past auctions, Elena figures she should expect to make about \$175 on the TV and pay about \$23 for the toaster oven. In total, how much should she expect to make or spend?\footnote{$E(X-Y) = E(X) - E(Y) = 175 - 23 = \$152$. She should expect to make about \$152.}
\end{exercise}

\begin{exercise} \label{explainWhyThereIsUncertaintyInTheSum}
Would you be surprised if John's weekly commute wasn't exactly 90 minutes or if Elena didn't make exactly \$152? Explain.\footnote{No, since there is probably some variability. For example, the traffic will vary from one day to next, and auction prices will vary depending on the quality of the merchandise and the interest of the attendees.}
\end{exercise}

Two important concepts concerning combinations of random variables have so far been introduced. First, a final value can sometimes be described as the sum of its parts in an equation. Second, intuition suggests that putting the individual average values into this equation gives the average value we would expect in total. This second point needs clarification -- it is guaranteed to be true in what are called \emph{linear combinations of random variables}.

A \term{linear combination} of two random variables $X$ and $Y$ is a fancy phrase to describe a combination
$$ aX + bY$$
where $a$ and $b$ are some fixed and known numbers. For John's commute time, there were five random variables -- one for each work day -- and each random variable could be written as having a fixed coefficient of 1:
$$ 1X_1 + 1 X_2 + 1 X_3 + 1 X_4 + 1 X_5 $$
For Elena's net gain or loss, the $X$ random variable had a coefficient of +1 and the $Y$ random variable had a coefficient of -1.

When considering the average of a linear combination of random variables, it is safe to plug in the mean of each random variable and then compute the final result. For a few examples of nonlinear combinations of random variables -- cases where we cannot simply plug in the means -- see the footnote.\footnote{If $X$ and $Y$ are random variables, consider the following combinations: $X^{1+Y}$, $X\times Y$, $X/Y$. In such cases, plugging in the average value for each random variable and computing the result will not generally lead to an accurate average value for the end result.}

\begin{termBox}{\tBoxTitle{Linear combinations of random variables and the average result}
If $X$ and $Y$ are random variables, then a linear combination of the random variables is given by $aX + bY$, where $a$ and $b$ are some fixed numbers. To compute the average value of a linear combination of random variables, plug in the average of each individual random variable and compute the result:
\begin{align*}
E(aX+bY) = a\times E(X) + b\times E(Y)
\end{align*}
Recall that the expected value is the same as the mean, i.e. $E(X) = \mu_{\mbox{\tiny\itshape X}}$.}
\end{termBox}

\begin{example}{Leonard has invested \$6000 in Google Inc. (stock ticker: GOOG) and \$2000 in Exxon Mobil Corp. (XOM). If $X$ represents the change in Google's stock next month and $Y$ represents the change in Exxon Mobil stock next month, write an equation that describes how much money will be made or lost in Leonard's stocks for the month.}
For simplicity, we will suppose $X$ and $Y$ are not in percents but are in decimal form (e.g. if Google's stock increases 1\%, then $X=0.01$; or if it loses 1\%, then $X=-0.01$). Then we can write an equation for Leonard's gain as
\begin{align*}
\$6000\times X + \$2000\times Y
\end{align*}
If we plug in the change in the stock value for $X$ and $Y$, this equation gives the change in value of Leonard's stock portfolio for the month. A positive value represents a gain, and a negative value represents a loss.
\end{example}

\begin{exercise}\label{expectedChangeInLeonardsStockPortfolio}
Suppose Google and Exxon Mobil stocks have recently been rising 2.1\% and 0.4\% per month, respectively. Compute the expected change in Leonard's stock portfolio for next month.\footnote{$E(\$6000\times X + \$2000\times Y) = \$6000\times 0.021 + \$2000\times 0.004 = \$134$.}
% library(stockPortfolio); gr <- getReturns(c("GOOG", "XOM"), start="2006-01-01"); gr
\end{exercise}

\begin{exercise}
You should have found that Leonard expects a positive gain in Guided Practice~\ref{expectedChangeInLeonardsStockPortfolio}. However, would you be surprised if he actually had a loss this month?\footnote{No. While stocks tend to rise over time, they are often volatile in the short term.}
\end{exercise}

\subsection{Variability in linear combinations of random variables}

Quantifying the average outcome from a linear combination of random variables is helpful, but it is also important to have some sense of the uncertainty associated with the total outcome of that combination of random variables. The expected net gain or loss of Leonard's stock portfolio was considered in Guided Practice~\ref{expectedChangeInLeonardsStockPortfolio}. However, there was no quantitative discussion of the volatility of this portfolio. For instance, while the average monthly gain might be about \$134 according to the data, that gain is not guaranteed. Figure~\ref{changeInLeonardsStockPortfolioFor36Months} shows the monthly changes in a portfolio like Leonard's during the 36 months from 2009 to 2011. The gains and losses vary widely, and quantifying these fluctuations is important when investing in stocks.

\begin{figure}[ht]
\centering
\includegraphics[width=0.65\textwidth]{ch_probability/figures/changeInLeonardsStockPortfolioFor36Months/changeInLeonardsStockPortfolioFor36Months}
\caption{The change in a portfolio like Leonard's for the 36 months from 2009 to 2011, where \$6000 is in Google's stock and \$2000 is in Exxon Mobil's.}
\label{changeInLeonardsStockPortfolioFor36Months}
\end{figure}

Just as we have done in many previous cases, we use the variance and standard deviation to describe the uncertainty associated with Leonard's monthly returns. To do so, the standard deviations and variances of each stock's monthly return will be useful, and these are shown in Table~\ref{sumStatOfGOOGXOM}. The stocks' returns are nearly independent.

\begin{table}
\centering
\begin{tabular}{lrrr}
\hline
	& Mean ($\bar{x}$) & Standard deviation ($s$) & Variance ($s^2$) \\
\hline
GOOG & 0.0210	& 0.0846					&	0.0072	\\
XOM & 0.0038		& 0.0519					&	0.0027	\\
\hline
\end{tabular}
\caption{The mean, standard deviation, and variance of the GOOG and XOM stocks. These statistics were estimated from historical stock data, so notation used for sample statistics has been used.}
\label{sumStatOfGOOGXOM}
\end{table}

We want to describe the uncertainty of Leonard's monthly returns by finding the standard deviation of the return on his combined portfolio. First, we note that the variance of a sum has a nice property: the variance of a sum is the sum of the variances. That~is, if X and Y are independent random variables:
\begin{align*}
Var(X + Y) &= Var(X) + Var(Y)
\end{align*}
Because the standard deviation is the square root of the variance, we can rewrite this equation using standard deviations:
\begin{align*}
(SD_{X + Y})^2 = (SD_X)^2 + (SD_Y)^2
\end{align*}
This equation might remind you of a theorem from geometry:  $c^2 = a^2 + b^2$. The equation for the standard deviation of the sum of two independent random variables looks analogous to the Pythagorean Theorem. Just as the Pythagorean Theorem only holds for right triangles, this equation only holds when X and Y are \emph{independent}.\footnote{Another word for independent is orthogonal, meaning right angle!  When X and Y are dependent, the equation for $SD_{X+Y}$ becomes analogous to the law of cosines.}

\begin{termBox}{\tBoxTitle{Standard Deviation of the sum  and difference of random variables}
If X and Y are \emph{independent} random variables:
\begin{align*}
SD_{X + Y} = SD_{X - Y} = \sqrt{(SD_X)^2 + (SD_Y)^2}
\end{align*}
}
\end{termBox}

Because $SD_Y$ = $SD_{-Y}$, the standard deviation of the difference of two variables equals the standard deviation of the sum of two variables. This property holds for more than two variables as well. For example, if X, Y, and Z are independent random variables:
\begin{align}
SD_{X + Y + Z} = SD_{X - Y - Z}  = \sqrt{(SD_X)^2 + (SD_Y)^2 + (SD_Z)^2}
\end{align}

If we need the standard deviation of a linear combination of independent variables, such as $aX + bY$, we can consider $aX$ and $bY$ as two new variables. Recall that multiplying all of the values of variable by a positive constant multiplies the standard deviation by that constant. Thus, $SD_{aX}$ =  $a \times SD_X$ and $SD_{bY}$ =  $b \times SD_Y$. It follows that:
\begin{align*}
SD_{aX + bY} = \sqrt{(a \times SD_X)^2 + (b \times SD_Y)^2}
\end{align*}
This equation can be used to compute the standard deviation of Leonard's monthly return. Recall that Leonard has \$6,000 in Google stock and \$2,000 in Exxon Mobil's stock. From Table~\ref{sumStatOfGOOGXOM}, the standard deviation of Google stock is 0.0846 and the standard deviation of Exxon Mobile stock is 0.0519.
\begin{align*}
SD_{6000X + 2000Y}
	&= \sqrt{(6000\times SD_X)^2 + (2000\times SD_Y)^2} \\
	&= \sqrt{(6000\times 0.0846)^2 + (4000\times .0519)^2} \\
	&= \sqrt{270,000} \\
	&= 520
\end{align*}
The standard deviation of the total is \$520. While an average monthly return of \$134 on an \$8000 investment is nothing to scoff at, the monthly returns are so volatile that Leonard should not expect this income to be very stable.

\begin{termBox}{\tBoxTitle{Standard Deviation of linear combinations of random variables}
To find the standard deviation of a linear combination of random variables, we first consider $aX$ and $bY$ separately. We find the standard deviation of each, and then we apply the equation for the standard deviation of the sum of two variables:
\begin{align*}
SD_{aX + bY} = \sqrt{(a\times SD_X)^2 + (b\times SD_Y)^2}
\end{align*}
This equation is valid as long as the random variables $X$ and $Y$ are \emph{independent} of each other.}
\end{termBox}

\begin{example}{Suppose John's daily commute has a standard deviation of 4 minutes. What is the uncertainty in his total commute time for the week?} \label{sdOfJohnsCommuteWeeklyTime}
The expression for John's commute time is
\begin{align*}
X_1 + X_2 + X_3 + X_4 + X_5
\end{align*}
Each coefficient is 1, so the standard deviation of the total weekly commute time is
\begin{align*}
\text{SD}&= \sqrt{(1 \times 4)^2 + (1 \times 4)^2 + (1 \times 4)^2 + (1 \times 4)^2 + (1 \times 4)^2} \\
&= \sqrt{5\times (4)^2} \\
&= 8.94
\end{align*}
The standard deviation for John's weekly work commute time is about 9 minutes.
\end{example}

\begin{exercise}
The computation in Example~\ref{sdOfJohnsCommuteWeeklyTime} relied on an important assumption: the commute time for each day is independent of the time on other days of that week. Do you think this is valid? Explain.\footnote{One concern is whether traffic patterns tend to have a weekly cycle (e.g. Fridays may be worse than other days). If that is the case, and John drives, then the assumption is probably not reasonable. However, if John walks to work, then his commute is probably not affected by any weekly traffic cycle.}
\end{exercise}

\begin{exercise}\label{elenaIsSellingATVAndBuyingAToasterOvenAtAnAuctionVariability}
Consider Elena's two auctions from Guided Practice~\ref{elenaIsSellingATVAndBuyingAToasterOvenAtAnAuction} on page~\pageref{elenaIsSellingATVAndBuyingAToasterOvenAtAnAuction}. Suppose these auctions are approximately independent and the variability in auction prices associated with the TV and toaster oven can be described using standard deviations of \$25 and \$8. Compute the standard deviation of Elena's net gain.\footnote{The equation for Elena can be written as:  $(1)\times X + (-1)\times Y$.

To find the SD of this new variable we do:
\begin{align*}
SD_{(1)\times X + (-1)\times Y} = \sqrt{(1\times SD_X)^2 + (-1\times SD_Y)^2 = (1\times 25)^2 + (-1\times 8)^2} = 26.25
\end{align*}
The SD is about \$26.25.}
\end{exercise}

Consider again Guided Practice~\ref{elenaIsSellingATVAndBuyingAToasterOvenAtAnAuctionVariability}. The negative coefficient for $Y$ in the linear combination was eliminated when we squared the coefficients. This generally holds true: negatives in a linear combination will have no impact on the variability computed for a linear combination, but they do impact the expected value computations.

\index{random variable|)}

%_________________
\section{Continuous distributions}
\label{contDist}

\index{data!FCID|(}
\index{hollow histogram|(}
\begin{example}{Figure~\ref{fdicHistograms} shows a few different hollow histograms of the variable \var{height} for 3 million US adults from the mid-90's.\footnote{This sample can be considered a simple random sample from the US population. It relies on the USDA Food Commodity Intake Database.} How does changing the number of bins allow you to make different interpretations of the data?}\label{usHeights}
Adding more bins provides greater detail. This sample is extremely large, which is why much smaller bins still work well. Usually we do not use so many bins with smaller sample sizes since small counts per bin mean the bin heights are very volatile.
\end{example}

\begin{figure}[ht]
\centering
\includegraphics[width=\textwidth]{ch_probability/figures/fdicHistograms/fdicHistograms}
\caption{Four hollow histograms of US adults heights with varying bin widths.}
\label{fdicHistograms}
\end{figure}

\begin{example}{What proportion of the sample is between \resp{180} cm and \resp{185} cm tall (about 5'11" to 6'1")?}\label{contDistProb}
We can add up the heights of the bins in the range \resp{180} cm and \resp{185} and divide by the sample size. For instance, this can be done with the two shaded bins shown in Figure~\ref{usHeightsHist180185}. The two bins in this region have counts of 195,307 and 156,239 people, resulting in the following estimate of the probability:
\begin{eqnarray*}
\frac{195307+156239}{\text{3,000,000}} = 0.1172
\end{eqnarray*}
This fraction is the same as the proportion of the histogram's area that falls in the range \resp{180} to \resp{185} cm.
\end{example}

\begin{figure}
\centering
\includegraphics[width=0.95\textwidth]{ch_probability/figures/usHeightsHist180185/usHeightsHist180185}
\caption{A histogram with bin sizes of 2.5 cm. The shaded region represents individuals with heights between \resp{180} and \resp{185} cm.}
\label{usHeightsHist180185}
\end{figure}

\subsection{From histograms to continuous distributions}

Examine the transition from a boxy hollow histogram in the top-left of Figure~\ref{fdicHistograms} to the much smoother plot in the lower-right. In this last plot, the bins are so slim that the hollow histogram is starting to resemble a smooth curve. This suggests the population height as a \emph{continuous} numerical variable might best be explained by a curve that represents the outline of extremely slim bins.

This smooth curve represents a \term{probability density function} (also called a \term{density} or \term{distribution}), and such a curve is shown in Figure~\ref{fdicHeightContDist} overlaid on a histogram of the sample. A density has a special property: the total area under the density's curve is 1.

\begin{figure}[tbh]
\centering
\includegraphics[width=0.87\textwidth]{ch_probability/figures/fdicHeightContDist/fdicHeightContDist}
\caption{The continuous probability distribution of heights for US adults.}
\label{fdicHeightContDist}
\end{figure}

\index{hollow histogram|)}


\subsection{Probabilities from continuous distributions}

We computed the proportion of individuals with heights \resp{180} to \resp{185} cm in Example~\ref{contDistProb} as a fraction:
\begin{eqnarray*}
\frac{\text{number of people between \resp{180} and \resp{185}}}{\text{total sample size}}
\end{eqnarray*}
We found the number of people with heights between \resp{180} and \resp{185} cm by determining the fraction of the histogram's area in this region. Similarly, we can use the area in the shaded region under the curve to find a probability (with the help of a computer):
\begin{eqnarray*}
P(\text{\var{height} between \resp{180} and \resp{185}})
	= \text{area between \resp{180} and \resp{185}}
	= 0.1157
\end{eqnarray*}
The probability that a randomly selected person is between \resp{180} and \resp{185} cm is 0.1157. This is very close to the estimate from Example~\ref{contDistProb}: 0.1172.

\begin{figure}
\centering
\includegraphics[width=0.8\textwidth]{ch_probability/figures/fdicHeightContDistFilled/fdicHeightContDistFilled}
\caption{Density for heights in the US adult population with the area between 180 and 185 cm shaded. Compare this plot with Figure~\ref{usHeightsHist180185}.}
\label{fdicHeightContDistFilled}
\end{figure}

\begin{exercise}
Three US adults are randomly selected. The probability a single adult is between \resp{180} and \resp{185} cm is 0.1157.\footnote{Brief answers: (a) $0.1157 \times 0.1157 \times 0.1157 = 0.0015$. (b) $(1-0.1157)^3 = 0.692$} \vspace{-1.5mm}
\begin{enumerate}
\setlength{\itemsep}{0mm}
\item[(a)] What is the probability that all three are between \resp{180} and \resp{185} cm tall?
\item[(b)] What is the probability that none are between \resp{180} and \resp{185} cm?
\end{enumerate}
\end{exercise}

\begin{example}{What is the probability that a randomly selected person is \textbf{exactly} \resp{180}~cm? Assume you can measure perfectly.}
\label{probabilityOfExactly180cm}
This probability is zero. A person might be close to \resp{180} cm, but not exactly \resp{180} cm tall. This also makes sense with the definition of probability as area; there is no area captured between \resp{180}~cm and \resp{180}~cm.
\end{example}

\begin{exercise}
Suppose a person's height is rounded to the nearest centimeter. Is there a chance that a random person's \textbf{measured} height will be \resp{180}~cm?\footnote{This has positive probability. Anyone between \resp{179.5} cm and \resp{180.5} cm will have a \emph{measured} height of \resp{180} cm. This is probably a more realistic scenario to encounter in practice versus Example~\ref{probabilityOfExactly180cm}.}
\end{exercise}

\index{data!FCID|)}



